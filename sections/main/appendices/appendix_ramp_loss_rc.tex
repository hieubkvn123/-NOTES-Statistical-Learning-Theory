\newpage
\subsection{Rademacher Complexity of the ramp loss}
\label{sec:rad_complexity_of_ramp_loss}

\subsubsection{Problem statement}
\textbf{Problem} : Consider the multi-class classification problem with $K$ labels ($K\ge2$). Given the following function class
\begin{align*}
    W  &= \biggCurl{
        \vecbf{w}\in \R^{K\times d} \Bigg| \|\vecbf{w}\|_F \le R
    } \\
    \F &= \biggCurl{
        f_{\vecbf{w}}:\R^d \to \R^{K} \Bigg| f_\vecbf{w}(\vecbf{x}) = \vecbf{wx}; \vecbf{w}\in W, \|x\|_2\le 1
    }
\end{align*}

\noindent and consider the following loss function:
\begin{align*}
    l_r(\vecbf{z}) &= \begin{cases}
        0 & \text{if } \vecbf{z} \ge r \\
        1 - \vecbf{z}/r &\text{if } \vecbf{z} \in (0,r) \\ 
        1 &\text{if } \vecbf{z} \le 0
    \end{cases} \\
\end{align*}

\noindent Derive the bound for the Rademacher complexity of the loss function class $\RC_n(\mathcal{L}_r)$ where:
\begin{align*}
    \mathcal{L}_r = \biggCurl{
        (x, y)\mapsto l_r\bigRound{ f_\vecbf{w}(x)_y - \max_{k\ne y} f_\vecbf{w}(x)_k } \Bigg| f_{\vecbf{w}} \in \F 
    }
\end{align*}

\subsubsection{Approach 1 : Using covering number}
\noindent\textbf{Overview} : Let $L_\vecbf{w} \in \mathcal{L}_r$ where
\begin{align*}
    L_\vecbf{w}(x,y) =  l_r\bigRound{ f_\vecbf{w}(x)_y - \max_{k\ne y} f_\vecbf{w}(x)_k }
\end{align*}


\noindent and $S=\bigCurl{(x_i, y_i)}_{i=1}^n$ be a sample dataset. Define the following norm:
\begin{align*}
    \|L_\vecbf{w}\|_2 = \sqrt{
        \frac{1}{n} \sum_{i=1}^n L_\vecbf{w}(x_i, y_i)^2
    }
\end{align*}

\noindent We have to find an $\epsilon$-covering with respect to $\|.\|_2$ norm, denoted as $\mathcal{C}_\epsilon({\mathcal{L}_r}_{|_S}, \|.\|_2)$. Meaning, $\forall L_\vecbf{w}\in\mathcal{L}_r$, $\exists L_\vecbf{\bar w} \in \mathcal{C}_\epsilon({\mathcal{L}_r}_{|_S}, \|.\|_2)$ such that:
\begin{align*}
    \|L_\vecbf{w} - L_\vecbf{\bar w}\|_2 = \sqrt{
        \frac{1}{n} \sum_{i=1}^n \bigRound{
            L_\vecbf{w}(x_i, y_i) - L_\vecbf{\bar w}(x_i, y_i)
        }^2
    } \le \epsilon
\end{align*}

\noindent\textbf{1. Bounding the covering number of $\mathcal{L}_r$} : Let $\vecbf{w}, \vecbf{\bar w}\in W$. For any pair $(x_i, y_i) \in S$, we have:
\begin{align*}
    \bigAbs{L_\vecbf{w}(x_i, y_i) - L_\vecbf{\bar w}(x_i, y_i)}
        &= \bigAbs{
            l_r\bigRound{ f_\vecbf{w}(x_i)_{y_i} - \max_{k\ne y_i} f_\vecbf{w}(x_i)_k } - l_r\bigRound{ f_\vecbf{\bar w}(x_i)_{y_i} - \max_{k\ne y_i} f_\vecbf{\bar w}(x_i)_k }
        } \\
        &\le \frac{1}{r}\bigAbs{
            \bigRound{f_\vecbf{w}(x_i)_{y_i} - f_\vecbf{\bar w}(x_i)_{y_i}} + \bigRound{\max_{k\ne y_i} f_\vecbf{\bar w}(x_i)_k - \max_{k\ne y_i} f_\vecbf{w}(x_i)_k}
        }\ \ \ (l_r \text{ is } 1/r-\text{Lipchitz}) \\
        &\le \frac{1}{r}\bigAbs{
            \bigRound{f_\vecbf{w}(x_i)_{y_i} - f_\vecbf{\bar w}(x_i)_{y_i}} + \bigRound{\max_{k\ne y_i} \bigCurl{f_\vecbf{\bar w}(x_i)_k - f_\vecbf{w}(x_i)_k}}
        } \\
        &\le \frac{1}{r}\bigAbs{f_\vecbf{w}(x_i)_{y_i} - f_\vecbf{\bar w}(x_i)_{y_i}} + \frac{1}{r}\max_{k\ne y_i}\bigAbs{f_{\vecbf{\bar w}}(x_i)_k - f_\vecbf{w}(x_i)_k} \\
        &\le \frac{2}{r}\sup_{\substack{x_i \in S \\ j \in \{1, \dots, K\}}}\bigAbs{
            f_\vecbf{w}^{(j)}(x_i) - f^{(j)}_\vecbf{\bar w}(x_i)
        } \\
        &= \frac{2}{r}\sup_{j\in\{1, \dots, K\}} \|f^{(j)}_\vecbf{w} - f^{(j)}_\vecbf{\bar w}\|_\infty \\
        &= \frac{2}{r}\max\bigCurl{\bigAbs{(\vecbf{w}_j - \vecbf{\bar w}_j)x_i}}_{\substack{i\in\{1, \dots, n\}, \\j\in\{1, \dots, K\}}} \ \ \ (1)
\end{align*}

\noindent Now, define the following class of functions:
\begin{align*}
    \Hf &= \bigCurl{
        h:\R^{Kd} \to \R : h(\vecbf{x}) = \beta \vecbf{x}, \beta \in \mathcal{B}
    } \\
    \mathcal{B} &= \bigCurl{
        \beta \in \R^{1\times Kd} : \beta = \begin{pmatrix}
            \vecbf{w}_1 & \dots & \vecbf{w}_K
        \end{pmatrix}, \ \vecbf{w} \in W
    }
\end{align*}

\noindent Basically, we construct the set of parameters $\beta\in \R^{1\times Kd}$ by concatenating rows of vectors $\vecbf{w}\in W$ horizontally. By the definition of $W$, we have that $\|\beta\|_2\le R, \ \forall \beta\in\mathcal{B}$.

\noindent\newline\newline For any sample $S=\{x_i\}_{i=1}^n$ of vectors $x_i\in \R^d$, construct a new sample $S'$ of $\R^{Kd}$ vectors by creating $K$ vectors from each $x_i\in S$:
\begin{align*}
    x_i \longrightarrow \begin{cases}
        x_{i, 1} &= \begin{pmatrix} x_i & \vecbf{0} & \vecbf{0} & \dots & \vecbf{0} \end{pmatrix}^T \\
        x_{i, 2} &= \begin{pmatrix} \vecbf{0} & x_i & \vecbf{0} & \dots & \vecbf{0} \end{pmatrix}^T \\
        &\vdots \\
        x_{i, K} &=\begin{pmatrix} \vecbf{0} & \vecbf{0} & \vecbf{0} & \dots & x_i \end{pmatrix}^T
    \end{cases}
\end{align*}

\noindent For all $i\in \{1, \dots, n\}$ and $j\in\{1, \dots, K\}$, we have $\|x_{i, j}\|_2 \le 1$. Hence, applying Zhang's theorem 4 \cite{article:tong_zhang} on $\Hf_{|S'}$, we have:
\begin{align*}
    \log_2\mathcal{N}\bigRound{\Hf_{|S'}, \epsilon, \|.\|_\infty} \le \frac{36R^2}{\epsilon^2}\log_2\biggRound{
        \frac{8RnK}{\epsilon} + 6nK + 1
    } \le \frac{36R^2}{\epsilon^2}\log_2\biggRound{\biggRound{
        \frac{8R}{\epsilon} + 7}nK
    } 
\end{align*}

\noindent In other words, if we denote $\mathcal{C}_\epsilon(\Hf_{|S'}, \|.\|_\infty)$ as the minimum $\epsilon$-covering for $\Hf_{|S'}$ with respect to the $\|.\|_\infty$ norm, we have:
\begin{align*}
    \bigAbs{
        \mathcal{C}_\epsilon(\Hf_{|S'}, \|.\|_\infty)
    } \le \frac{36R^2}{\epsilon^2}\log_2\biggRound{\biggRound{
        \frac{8R}{\epsilon} + 7}nK
    }
\end{align*}

\noindent And for any $h\in \Hf$ parameterized by $\beta=\begin{pmatrix} \vecbf{w_1} &\dots &\vecbf{w_K}\end{pmatrix}$, there exists $\bar h \in \mathcal{C}_\epsilon(\Hf_{|S'}, \|.\|_\infty)$, parameterized by $\bar\beta=\begin{pmatrix} \vecbf{\bar w_1} &\dots &\vecbf{\bar w_K}\end{pmatrix}$, such that $\|h - \bar h\|_\infty \le \epsilon$. Hence, we have:
\begin{align*}
    \|h - \bar h\|_\infty &= \sup_{x_{i, j} \in S'} \bigAbs{h(x_{i,j}) - \bar h(x_{i,j})} \\
        &= \sup_{x_{i,j}\in S'}\bigAbs{(\beta - \bar\beta)x_{i,j}} \\
        &= \max_{\substack{x_i\in S \\ j\in\{1, \dots, K\}}}\bigAbs{(\vecbf{w}_j - \vecbf{\bar w}_j)x_i} < \epsilon \ \ \ (2)
\end{align*}

\noindent From $(1)$ and $(2)$, we have:
\begin{align*}
    \log_2\mathcal{N}\bigRound{{\mathcal{L}_r}_{|S}, \epsilon, \|.\|_2} 
    &\le \log_2\mathcal{N}\bigRound{\Hf_{|S'}, r\epsilon/2, \|.\|_\infty}
    \le \frac{144R^2}{r^2\epsilon^2}\log_2\biggRound{
        \biggRound{
            \frac{16R}{r\epsilon} + 7
        }nK
    }
\end{align*}

\noindent\newline\textbf{2. Dudley's Entropy Integral} : Using theorem \ref{thm:dudley_entropy_integral}, we have:
\begin{align*}
    \ERC_S(\mathcal{L}_r) \le 4\alpha + \frac{12}{\sqrt n}\int_{\alpha}^1 \sqrt{\log\mathcal{N}\bigRound{\mathcal{L}_r, \epsilon, \|.\|_2}}d\epsilon, \ \alpha > 0 
\end{align*}

\noindent Note that we made use of the fact that $\sup_{\vecbf{z} \in \R} l_r(\vecbf{z}) = 1$. Hence, $\sup_{L_\vecbf{w}\in\mathcal{L}_r} \|L_\vecbf{w}\|_2 = 1$. Hence, the upper limit of the integral is $1$. From the above covering number bound, we have:
\begin{align*}
    \ERC_S(\mathcal{L}_r) 
        &\le 4\alpha + \frac{144R}{r\sqrt n}\int_{\alpha}^1 \frac{1}{\epsilon}\sqrt{\log\biggRound{
            \biggRound{\frac{16R}{r\epsilon}+7
        }nK}}d\epsilon \\
        &< 4\alpha + \frac{144R}{r\sqrt n} \cdot  \sqrt{
            \log\biggRound{\biggRound{
                \frac{16R}{r\alpha} + 7
            }nK}
        }\int_\alpha^1 \frac{1}{\epsilon}d\epsilon \\
        &= 4\alpha + \frac{144R}{r\sqrt n} \cdot  \sqrt{
            \log\biggRound{\biggRound{
                \frac{16R}{r\alpha} + 7
            }nK}
        }\bigRound{-\log\alpha}
\end{align*}

\noindent Setting $\alpha = \frac{36R}{r\sqrt n}$, we have the following upperbound:
\begin{align*}
    \ERC_S(\mathcal{L}_r) \le \frac{144R}{r\sqrt{n}}\biggSquare{1 + \sqrt{
        \log\biggRound{\biggRound{
            \frac{4}{9}\sqrt{n} + 7
        }nK}
    }\log\biggRound{\frac{r\sqrt n}{36R}}}
\end{align*}

\noindent Therefore, we have:
\begin{align*}
    \RC_n(\mathcal{L}_r) \le \tilde O\biggRound{\frac{R}{r\sqrt{n}}}
\end{align*}

\subsubsection{Approach 2 : Using contraction inequality}
\noindent\textbf{1. Find Rademacher complexity of each output component} : First, we bound the covering number of each component of the output then work out the Rademacher complexity.\newline
\noindent For any $f_\vecbf{w}\in\F$, we have:
\begin{align*}
    f_\vecbf{w} = \begin{pmatrix}
        f_\vecbf{w}^{(1)}(x) \\
        \vdots \\
        f_\vecbf{w}^{(K)}(x)
    \end{pmatrix}
    = 
    \begin{pmatrix}
        \vecbf{w}_{1}x \\
        \vdots \\
        \vecbf{w}_{K}x
    \end{pmatrix}
\end{align*}

\noindent Where $\vecbf{w}_j$ is the $j^{th}$ row of $\vecbf{w}$ for $j=\{1, \dots, K\}$. Define the following classes:
\begin{align*}
    \F_j = \bigCurl{f_\vecbf{w}^{(j)} : f_\vecbf{w} \in \F} = \bigCurl{\vecbf{x} \mapsto \vecbf{w}_j\vecbf{x} : \vecbf{w} \in W}, \ \ j \in \{1, \dots, K\}
\end{align*}

\noindent\newline Since for any $\vecbf{w}\in W$, we have $\|\vecbf{w}\|_F\le R$. Therefore, for all $j\in\{1, \dots, K\}$, we have $\|\vecbf{w}_j\|_2\le R$. Hence, by theorem 3 in \cite{article:tong_zhang}, we have:
\begin{align*}
    \log_2\mathcal{N}\bigRound{\F_j, \epsilon, \|.\|_2} \le \Bigg\lceil \frac{R^2}{\epsilon^2} \Bigg\rceil \log_2(2d+1) \le 2\frac{R^2}{\epsilon^2}\log_2(2d+1)
\end{align*}

\noindent Assuming that $R\ge\epsilon$. Hence, by Dudley's \ref{thm:dudley_entropy_integral}, we have:
\begin{align*}
    \ERC_S(\F_j) &\le 4\alpha + \frac{12R}{\sqrt{n}}\sqrt{\log D} \int_\alpha^R \frac{1}{\epsilon}d\epsilon, \ \ \ D = (2d+1)^2 \\
    &= 4\alpha + \frac{12R}{\sqrt{n}}\sqrt{\log D}\log\frac{R}{\alpha} \\
    &= \frac{12R}{\sqrt n} \biggRound{1 + \sqrt{\log D}\log\frac{\sqrt n}{3}}, \ \ \ \text{Setting } \alpha = \frac{3R}{\sqrt n} \\
    \implies
    \RC_n(\F_j) &\le \frac{12R}{\sqrt n} \biggRound{1 + \sqrt{\log D}\log\frac{\sqrt n}{3}} = \tilde O\biggRound{\frac{R}{\sqrt{n}}}
\end{align*}

\noindent\textbf{2. Using $l_\infty$ contraction inequality} : By theorem 1 in \cite{article:foster}, if we have $\phi_1, \dots, \phi_n$, where $\phi_i:\R^K \to \R$, being $L$-Lipchitz with respect to the $l_\infty$ norm, meaning $|\phi_j(x)-\phi_j(y)| \le L\cdot\|x-y\|_\infty$. We have:
\begin{align*}
    \frac{1}{n}\E_\sigma\biggSquare{
        \sup_{f_\vecbf{w}\in \F}\sum_{i=1}^n \sigma_i \phi_i(f_\vecbf{w}(x_i))
    } \le \tilde O\bigRound{L\sqrt{K}} \cdot \max_{j\in\{1, \dots, K\}} \RC_n(\F_j)
\end{align*}

\noindent For any $z\in\R^K$, define the function $\psi_j(z)=z_j - \max_{k\ne j}z_k$ and for any $(x_i, y_i)\in S$, let $\phi_i(f_\vecbf{w}(x_i)) = (l_r\circ \psi_{y_i})(f_\vecbf{w}(x_i))$. We have:
\begin{align*}
    \ERC_S(\mathcal{L}_r) 
        &= \frac{1}{n}\E_\sigma\biggSquare{
            \sup_{f_\vecbf{w}\in \F} \sum_{i=1}^n \sigma_i l_r\bigRound{f_\vecbf{w}(x_i)_{y_i} - \max_{k\ne y_i} f_\vecbf{w}(x_i)_k}
        } \\
        &= \frac{1}{n}\E_\sigma\biggSquare{
            \sup_{f_\vecbf{w}\in \F} \sum_{i=1}^n \sigma_i \phi_i(f_\vecbf{w}(x_i))
        }
\end{align*}

\noindent Following similar arguments as $(1)$, we know that $l_r \circ \psi_{j}, \ \ j\in\{1, \dots, K\}$ is $2/r$-Lipchitz continuous with respect to the $l_\infty$ norm. Hence, we have:
\begin{align*}
    \ERC_S(\mathcal{L}_r) 
        &\le \tilde O\biggRound{
            \frac{2\sqrt{K}}{r}
        }\cdot \max_{j\in\{1, \dots, K \}} \RC_n(\F_j) \\
        &= \tilde O\biggRound{
           \frac{R\sqrt{K}}{r\sqrt n}
        }
\end{align*}

\subsubsection{Approach 3 : Stacking covering numbers}
\textbf{1. Stacking covering numbers} : By Zhang's theorem 4 \cite{article:tong_zhang}, we have:
\begin{align*}
    \log_2\mathcal{N}\bigRound{{\F_j}_{|S}, \epsilon, \|.\|_\infty} \le \frac{36R^2}{\epsilon^2}\log_2\biggRound{
        \frac{8Rn}{\epsilon} + 6n + 1
    }, \ \ j \in \{ 1, \dots, K \}
\end{align*}

\noindent Denote $C^{(j)}_\epsilon$ as the minimum (internal) $\epsilon$-cover of $\F_j=\bigCurl{\vecbf{x} \mapsto \vecbf{w}_j \vecbf{x}: \vecbf{w}\in W}$ with respect to the $\|.\|_\infty$ norm. Meaning, for any $f_\vecbf{w} \in \F$:
\begin{align*}
    \exists f_\vecbf{\bar w}^{(j)} \in C^{(j)}_\epsilon : \|f_\vecbf{w}^{(j)} - f_\vecbf{\bar w}^{(j)} \|_\infty = \sup_{i \in \{1, \dots, n \}}\bigAbs{f_\vecbf{w}^{(j)}(x_i) - f_\vecbf{\bar w}^{(j)}(x_i)} < \epsilon, \ \ \ \vecbf{\bar w}\in W
\end{align*}

\noindent Therefore, for any $f_\vecbf{w}\in \F$:
\begin{align*}
    \exists f_\vecbf{\bar w} \in C^{(1)}_\epsilon \times \dots \times C^{(K)}_\epsilon : \max_{j\in \{1, \dots, K \}} \|f_\vecbf{w}^{(j)} - f_\vecbf{\bar w}^{(j)}\|_\infty = \max_{
        \substack{ j \in \{1, \dots, K \}\\ i \in \{1, \dots, n\}}
    }\bigAbs{
        f_\vecbf{w}^{(j)}(x_i) - f_\vecbf{\bar w}^{(j)}(x_i) 
    } < \epsilon \ \ \ (3)
\end{align*}

\noindent However, note that even though $f_\vecbf{\bar w}\in C^{(1)}_\epsilon \times \dots \times C^{(K)}_\epsilon$, it does not necessarily mean $f_\vecbf{\bar w}\in\F$. Because in the worst case senario, we have:
\begin{align*}
    \|\vecbf{\bar w}\|_F &= \sqrt{\sum_{j=1}^K \|\vecbf{\bar w}_j\|_2^2} \le \sqrt{\sum_{j=1}^K R^2} = R\sqrt{K}
\end{align*}

\noindent Hence, $C^{(1)}_\epsilon \times \dots \times C^{(K)}_\epsilon$ is an external $\epsilon$-cover. From $(1)$, $(3)$ and lemma \ref{lem:external_internal_cover}, we have:
\begin{align*}
    \mathcal{N}\bigRound{{\mathcal{L}_r}_{|S}, \epsilon, \|.\|_2} 
        &\le \mathcal{N}^{ext}\bigRound{{\mathcal{L}_r}_{|S}, \epsilon/2, \|.\|_2} \\ 
        &\le \bigAbs{
            C^{(1)}_{r\epsilon/4} \times \dots \times C^{(K)}_{r\epsilon/4}
        } \\
        &\le \prod_{j=1}^K \bigAbs{C^{(j)}_{r\epsilon/4}} \\
    \implies
    \log \mathcal{N}\bigRound{{\mathcal{L}_r}_{|S}, \epsilon, \|.\|_2} 
        &\le \sum_{j=1}^K \log \bigAbs{C^{(j)}_{r\epsilon/4}} \\
        &\le \frac{576R^2K}{r^2\epsilon^2} \log\biggRound{
            \frac{32Rn}{r\epsilon} + 6n + 1
        } \\
        &\le \frac{576R^2K}{r^2\epsilon^2} \log\biggRound{
            \biggRound{\frac{32R}{r\epsilon} + 7}n
        }
\end{align*}


\noindent \textbf{2. Dudley's Entropy Integral} : By Dudley's \ref{thm:dudley_entropy_integral}, we have:
\begin{align*}
    \ERC_S(\mathcal{L}_r) &\le 4\alpha + \frac{12}{\sqrt n}\int_\alpha^1 \sqrt{\log \mathcal{N}\bigRound{{\mathcal{L}_r}_{|S}, \epsilon, \|.\|_2}}d\epsilon, \ \ \ \alpha > 0 \\
    &= 4\alpha + \frac{288R\sqrt K}{r\sqrt n} \int_\alpha^1 \frac{1}{\epsilon}\sqrt{
        \log\biggRound{
            \biggRound{\frac{32R}{r\epsilon} + 7}n
        } 
    }d\epsilon \\
    &< 4\alpha + \frac{288R\sqrt K}{r\sqrt n}\cdot\sqrt{
        \log\biggRound{
            \biggRound{\frac{32R}{r\alpha} + 7}n
        } 
    }\int_{\alpha}^1 \frac{1}{\epsilon}d\epsilon \\
    &= 4\alpha + \frac{288R\sqrt K}{r\sqrt n}\cdot\sqrt{
        \log\biggRound{
            \biggRound{\frac{32R}{r\alpha} + 7}n
        } 
    }\bigRound{-\log\alpha}
\end{align*}

\noindent Setting $\alpha = \frac{72R\sqrt{K}}{r\sqrt n}$, we have:
\begin{align*}
    \ERC_S(\mathcal{L}_r) &< \frac{288R\sqrt K}{r\sqrt n}\biggSquare{1 + \sqrt{
        \log\biggRound{
            \biggRound{\frac{4\sqrt n}{9\sqrt K} + 7}n
        } 
    }\log\biggRound{\frac{r\sqrt{n}}{72R\sqrt K}}} \\
    &= \tilde O\biggRound{
        \frac{R\sqrt K}{r\sqrt n}
    }
\end{align*}

\subsection{Important lemmas and theorems for \ref{sec:rad_complexity_of_ramp_loss}}
\subsubsection{$l_\infty$ Contraction Inequality}
\begin{theorem}{$l_\infty$ Contraction Inequality}{l_infty_contraction_inequality}
    Let $\F\subseteq\{f:\X \to \R^K\}$ and $\phi_1, \dots, \phi_n$ be $L$-Lipchitz with respect to the $l_\infty$ norm. We have:
    \begin{align*}
        \E_\sigma\biggSquare{
            \sup_{f\in\F}\sum_{i=1}^n \sigma_i\phi_i\bigRound{f(x_i)} 
        }\le \tilde O\bigRound{L\sqrt K}\cdot\max_{j\in\{1, \dots, K\}} \RC_n(\F_j)
    \end{align*}
\end{theorem}

\begin{proof*}[Theorem \ref{thm:l_infty_contraction_inequality}]
    The proof included in theorem 1 of \cite{article:foster} made use of the fat-shattering coefficients, which is not introduced in this note. Hence, it will not be rewritten here.
\end{proof*}


\subsubsection{External-internal $\epsilon$-covers}
\begin{lemma}{External-internal covering numbers}{external_internal_cover}
    Let $(X, \|.\|)$ be a normed space and consider a subspace $V\subset X$. Let $\epsilon > 0$ and denote $\mathcal{N}(V, \epsilon, \|.\|)$ as the (internal) covering number of $V$, $\mathcal{N}^{ext}(V, \epsilon, \|.\|)$ as the external covering number of $V$. We have:
    \begin{align*}
        \mathcal{N}\bigRound{V, \epsilon, \|.\|} \le \mathcal{N}^{ext}\bigRound{V, \epsilon/2, \|.\|}
    \end{align*} 
\end{lemma}

\begin{proof*}[Lemma \ref{lem:external_internal_cover}]
    Let $\mathcal{C}_{\epsilon/2}^{ext}\bigRound{V, \|.\|}=\bigCurl{v_1^{ext}, \dots, v_{N_0}^{ext}}$ be the minimal $\epsilon/2$-cover of $V$ with respect to the norm $\|.\|$ where $N_0 = \bigAbs{\mathcal{C}_{\epsilon/2}^{ext}\bigRound{V, \|.\|}} = \mathcal{N}^{ext}\bigRound{V, \epsilon/2, \|.\|}$. We have:
    \begin{align*}
        V \subseteq \bigcup_{i=1}^{N_0} \mathcal{B}_{\epsilon/2}(v_i^{ext})
    \end{align*}

    \noindent Where for $\epsilon>0$, $\mathcal{B}_{\epsilon}(x)$ is the $\epsilon$-ball centered around $x$. For every $v_i^{ext}\in \mathcal{C}_{\epsilon/2}^{ext}\bigRound{V, \|.\|}$, we have:
    \begin{align*}
        V \cap \mathcal{B}_{\epsilon/2}(v_i^{ext}) \ne \emptyset
    \end{align*}

    \noindent Otherwise, $v_i^{ext}$ is redundant which contradicts the fact that $\mathcal{C}_{\epsilon/2}^{ext}\bigRound{V, \|.\|}$ is a minimum external $\epsilon/2$-cover of $V$. Hence, for all $v_i^{ext}$, we have:
    \begin{align*}
        \exists v_i^{in} \in V \cap \mathcal{B}_{\epsilon/2}(v_i^{ext}) : \mathcal{B}_{\epsilon/2}(v_i^{ext}) \subset \mathcal{B}_{\epsilon}(v_i^{in})
    \end{align*}

    \noindent Therefore, we have:
    \begin{align*}
        V \subseteq \bigcup_{i=1}^{N_0} \mathcal{B}_{\epsilon/2}(v_i^{ext}) \subseteq \bigcup_{i=1}^{N_0}\mathcal{B}_\epsilon(v_i^{in})
    \end{align*}

    \noindent Notice that from the above, it is possible to cover $V$ with fewer than $N_0$ $\epsilon$-balls $\mathcal{B}_\epsilon(v_i^{in})$. Hence, we have:
    \begin{align*}
        \mathcal{N}\bigRound{V, \epsilon, \|.\|} \le N_0 = \mathcal{N}^{ext}\bigRound{V, \epsilon/2, \|.\|}
    \end{align*}
\end{proof*}
