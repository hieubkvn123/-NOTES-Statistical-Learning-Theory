\subsection{Rademacher Complexity bound for linear function classes}
In the following section, we will explore a simple exercise for bounding the Rademacher Complexity for linear function classes.

\subsubsection{Problem Statement}
\label{sec:rc_bound_lin_func_problem}

\textbf{Problem} : Let $\F$ be a linear function class defined as followed:
\begin{align*}
    \F = \bigCurl{
        f:\R^d \to \R \Big| f(x) = wx, \|w\|_2 \le R
    }
\end{align*}

\noindent Our objective is to prove the following Rademacher Complexity bound:
\begin{align*}
    \RC_n(\F) \le \tilde O \biggRound{
        \frac{R}{\sqrt{n}}
    }
\end{align*}

\noindent Before solving the above problem, we have to get familiar with the definition of \textbf{covering number} and some related lemmas.


\subsubsection{Covering Number}
\begin{definition}[$\epsilon$-Cover]
    Let $Q$ be a set. $\mathcal{C}$ is called an \textbf{$\epsilon$-cover} of $Q$ with respect to a metric $\rho$ if:
    \begin{align*}
        \forall v \in Q, \exists v'\in \mathcal{C} : \rho(v, v') \le \epsilon
    \end{align*}

    \noindent Basically, $\mathcal{C}$ is an \textbf{$\epsilon$-ball overlapping $Q$} (Figure \ref{fig:epsilon_cover}).
\end{definition}

\begin{figure}[ht]
    \centering
    \includegraphics[width=\textwidth]{figures/epsilon_cover.png}
    \caption{Examples of $\epsilon$-covers.}
    \label{fig:epsilon_cover}
\end{figure}

\begin{definition}[Covering Number of sets ($\mathcal{N}(Q, \epsilon, \rho$)]
    The \textbf{covering number} of a set $Q$ is defined as the minimum number of $\epsilon$-covers needed to completely contain $Q$:
    \begin{align*}
        \mathcal{N}(Q, \epsilon, \rho) = \text{minimum size of $\epsilon$-covers of $Q$ w.r.t $\rho$}
    \end{align*}

    \noindent Visual illustration of covering number is included in figure \ref{fig:covering_number}.
\end{definition}

\begin{figure}[ht]
    \centering
    \includegraphics[width=\textwidth]{figures/covering_number.png}
    \caption{Example of covering number. For $\F$, the covering number is 5. For $\F'$, the covering number is 10.}
    \label{fig:covering_number}
\end{figure}


\begin{definition}[Covering number of function class ($\mathcal{N}(\F, \epsilon, \rho)$)]
    Let $\F$ be a function class. Define the restriction of $\F$ to the observations $\{x_1, \dots, x_n\}$ as:
    \begin{align*}
        \F_{x_1, \dots, x_n} = \bigCurl{
            (f(x_1), \dots, f(x_n)) : f \in \F
        }
    \end{align*}

    \noindent Then, we can define the \textbf{Covering Number} for $\F$ as followed:
    \begin{align*}
        \mathcal{N}(\F, \epsilon, \rho) = \sup_{x_1, \dots, x_n} \mathcal{N}(\F_{x_1, \dots, x_n}, \epsilon, \rho)
    \end{align*}

    \noindent We can call $\mathcal{N}(\F_{x_1, \dots, x_n}, \epsilon, \rho)$ the \textbf{Empirical Covering Number}.
\end{definition}

\subsubsection{Bound on covering number of linear function class}
\begin{lemma}{Bound on $\mathcal{N}(\F, \epsilon, \|.\|_2)$ for linear functions}{covering_num_bound_lin_func}
    Let $\F$ be a linear function class defined as followed:
    \begin{align*}
        \F = \bigCurl{
            f:\R^d\to\R \Big| f(x) = wx, \|w\|_q \le a, \|x\|_p \le b
        }
    \end{align*}

    \noindent Where $p, q$ are Holder conjugates and $2\le p\le \infty$. Then, we have:
    \begin{align*}
        \log\mathcal{N}(\F, \epsilon, \|.\|_2) \le \Bigg\lceil \frac{a^2b^2}{\epsilon^2} \Bigg\rceil \log(2d+1)
    \end{align*}

    \noindent The proof of this lemma is discussed in Theorem 3 in \cite{article:tong_zhang}.
\end{lemma}

\begin{proof*}[Lemma \ref{lem:covering_num_bound_lin_func}]
    
\end{proof*}


\subsubsection{Dudley's Theorem}
\begin{theorem}{Dudley's Theorem}{dudley_theorem}
    If $\F$ is a function class from $\mathcal{Z}\to\R$ (where $\mathcal{Z}$ is a vector space), then:
    \begin{align*}
        \ERC_S(\F) \le 12\int_0^\infty\sqrt{
            \frac{\log \mathcal{N}(\F, \epsilon, \|.\|_2)}{n}
        }d\epsilon
    \end{align*}
\end{theorem}
\begin{proof*}[Theorem \ref{thm:dudley_theorem}]
    
\end{proof*}

\noindent\newline\textbf{Remark} : Using theorem \ref{thm:dudley_theorem}, we can translate the covering number to the Empirical Rademacher Complexity given that the covering number has some special formulation and the function class $\F$ is bounded in $[-1, 1]$. For example:
\begin{itemize}
    \item ${\bf (i)} \ \mathcal{N}(\F, \epsilon, \|.\|^2) \approx (1/\epsilon)^R$.

    \noindent Then we have $\log\mathcal{N}(\F, \epsilon, \|.\|^2) \approx R\log(1/\epsilon)$. Therefore,
    \begin{align*}
        \int_0^1 \sqrt{
            \frac{\log \mathcal{N}(\F, \epsilon, \|.\|_2)}{n}
        }d\epsilon = \int_0^1\sqrt{
            \frac{R\log(1/\epsilon)}{n}
        }d\epsilon \approx \sqrt{\frac{R}{n}}
    \end{align*}

    \item ${\bf (ii)} \ \mathcal{N}(\F, \epsilon, \|.\|^2) \approx a^{R/\epsilon}$.

    \noindent Then we have $\log\mathcal{N}(\F, \epsilon, \|.\|^2) \approx \frac{R}{\epsilon}\log a$. Therefore,
    \begin{align*}
        \int_0^1 \sqrt{
            \frac{\log \mathcal{N}(\F, \epsilon, \|.\|_2)}{n}
        }d\epsilon &\approx \int_0^1\sqrt{
            \frac{R}{n\epsilon}\log a 
        }d\epsilon \\
        &= \sqrt{
            \frac{R}{n}\log a 
        }\int_0^1 \sqrt{\frac{1}{\epsilon}}d\epsilon \\
        &= \tilde O\biggRound{ \sqrt{\frac{R}{n}} }
    \end{align*}

    \item ${\bf (iii)} \ \mathcal{N}(\F, \epsilon, \|.\|^2) \approx a^{R/\epsilon^2}$.
    
    \noindent Then, we have $\log \mathcal{N}(\F, \epsilon, \|.\|^2)\approx \frac{R}{\epsilon^2}\log a$. Therefore,
    \begin{align*}
        \int_0^1 \sqrt{
            \frac{\log \mathcal{N}(\F, \epsilon, \|.\|^2)}{n}
        }d\epsilon \approx \sqrt{
            \frac{R}{n}\log a 
        }\int_0^1 \frac{1}{\epsilon}d\epsilon = \infty
    \end{align*}
\end{itemize}

\subsubsection{Rademacher Complexity bound for linear functions class}
In this section, we finally solve the problem stated in section \ref{sec:rc_bound_lin_func_problem}. First, consider the following lemma, then prove the proposition \ref{prop:rc_bound_for_lin_func}:

\begin{theorem}{Dudley's Entropy Integral bound}{dudley_entropy_integral}
    Let $\F$ be a real-valued function class taking values in $[-1, 1]$ and assume that ${\bf 0}\in\F$. Then,
    \begin{align*}
        \ERC_S(\F) \le \inf_{\alpha>0} \biggRound{
            4\alpha + \frac{12}{\sqrt n}\int_\alpha^{\sqrt n} \sqrt{
                \log\mathcal{N}(\F, \epsilon,\|.\|_2)
            }d\epsilon
        }
    \end{align*}
\end{theorem}
\begin{proof*}[Theorem \ref{thm:dudley_entropy_integral}]
    
\end{proof*}


\begin{proposition}{Rademacher Complexity bound for linear function class}{rc_bound_for_lin_func}
    Given the following function class $\F$ whose range is bounded within $[-1, 1]$:
    \begin{align*}
        \F = \bigCurl{
            f:\R^d\to\R \Big| f(x) = wx, \|w\|_2 \le a, \|x\|_2 \le b
        }
    \end{align*}

    \noindent Then, we have the following bound for the Rademacher Complexity:
    \begin{align*}
        R_N(\F) \le \tilde O \biggRound{\frac{R}{\sqrt{n}}}, \ \ R = ab
    \end{align*}
\end{proposition}

\begin{proof*}[Proposition \ref{prop:rc_bound_for_lin_func}]
    \noindent From lemma \ref{lem:covering_num_bound_lin_func}, we have the following bound on the covering number of $\F$:
    \begin{align*}
        \log\mathcal{N}(\F, \epsilon, \|.\|_2) \le \Bigg\lceil \frac{R^2}{\epsilon^2} \Bigg\rceil \log(2d+1) < 2\frac{R^2}{\epsilon^2}\log(2d+1) = \frac{R^2}{\epsilon^2}\log(4d^2 + 4d + 1)
    \end{align*}

    \noindent The second inequality holds under the assumption that $R^2 > \epsilon^2$. Let $D = 4d^2 + 4d + 1$, we have:
    \begin{align*}
        \int_\alpha^{\sqrt n} \sqrt{\log\mathcal{N}(\F, \epsilon, \|.\|_2)} 
            &< R\sqrt{\log D}\int_\alpha^{\sqrt n} \frac{1}{\epsilon}d\epsilon \\
            &= R\sqrt{\log D}\bigRound{ \log\sqrt{n} - \log\alpha }
    \end{align*}

    \noindent Using theorem \ref{thm:dudley_entropy_integral}, we have:
    \begin{align*}
        \ERC_S(\F) 
            &\le \inf_{\alpha>0}\biggRound{
                4\alpha + \frac{12}{\sqrt n} \int_\alpha^{\sqrt n} \sqrt{\log\mathcal{N}(\F, \epsilon, \|.\|_2)} d\epsilon
            } \\
            &< \inf_{\alpha>0}\biggRound{
                4\alpha + \frac{12}{\sqrt n} R\sqrt{\log D} \bigRound{ \log\sqrt{n} - \log\alpha }
            }
    \end{align*}

    \noindent Letting $\alpha=\frac{3R}{\sqrt n}$, we have:
    \begin{align*}
        \ERC_S(\F) &< \frac{12R}{\sqrt n} + \frac{12R}{\sqrt n}\sqrt{\log D} \biggRound{ \log \sqrt{n} - \log\frac{3R}{\sqrt n} } \\
        &= \frac{12R}{\sqrt n} \biggSquare{
            1 + \sqrt{\log D}\log\frac{n}{3R}
        } \\
        &= \tilde O\biggRound{ \frac{R}{\sqrt n} }
    \end{align*}

    \noindent Since the right-hand-side does not depend on the sample $S$, we can just take expectation over the samples for both sides and we have:
    \begin{align*}
        \RC_n(\F) < \tilde O\biggRound{ \frac{R}{\sqrt n} }
    \end{align*}
\end{proof*}

