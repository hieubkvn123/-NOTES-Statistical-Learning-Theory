\begin{exercise}{}{exercise_5.1}
    Determine the sample complexity $N(\epsilon, \delta)$ for ERM over a class $\Hf$ with VC dimension $V_\Hf<\infty$.
\end{exercise}

\begin{solution*}[Exercise \ref{ex:exercise_5.1}]
    We have:
    \begin{align*}
        P\bigRound{R(\widehat{h_n}) - R_\Hf^* \ge \epsilon} 
        &\le P\biggRound{
            2\sup_{h\in\Hf}\bigAbs{ \widehat{R_n}(h) - R(h) } \ge \epsilon
        } \\
        &= P\biggRound{
            \sup_{h\in\Hf}\bigAbs{ \widehat{R_n}(h) - R(h) } \ge \frac{\epsilon}{2}
        } \\
        &\le 8S_\Hf(n)e^{-n\epsilon^2/128} \ \ \ (\text{Theorem } \ref{thm:udb_non_finite_h}) \\
        &\le 8(n+1)^{V_\Hf} e^{-n\epsilon^2/128} \ \ \ (\text{Corollary } \ref{coro:sauer_bound_on_shattering_coeff_I})
    \end{align*}

    \noindent Now let:
    \begin{align*}
        \delta &= 8(n+1)^{V_\Hf}e^{-n\epsilon^2/128} \\
        \implies
        \log\frac{\delta}{8} &= V_\Hf\log(n+1) - \frac{n\epsilon^2}{128}
        \\
        \implies
        N(\epsilon, \delta) &= \frac{128}{\epsilon^2} \biggRound{
            V_\Hf\log (n+1) - \log\frac{\delta}{8}
        }
    \end{align*}
\end{solution*}

\begin{exercise}{}{exercise_5.2}
    Show that the VC Theorem for sets implies the VC Theorem for classifiers. 

    \noindent\newline\textit{Hint : Consider the sets of the form $G'=G\times\{0\} \cup G^c \times \{1\} \subset \X\times\Y$.}
\end{exercise}

\begin{solution*}[Exercise \ref{ex:exercise_5.2}]
    Given an arbitrary class of classifiers $\Hf$. Define the following class of sets:
    \begin{align*}
        \mathcal{G} = \bigCurl{
            G_h \times \{0\} \cup G_h^c \times \{1\} : h \in \Hf
        }
    \end{align*}

    \noindent Where for a given $h\in\Hf$, we have:
    \begin{align*}
        G_h = \bigCurl{
            x \in \X : h(x) = 1
        }
    \end{align*}

    \noindent Let $P_{XY}$ be the density function over $\X\times\Y$. For any $G\in\mathcal{G}$, we have:
    \begin{align*}
        P_{XY}(G) &= \pi_0P_{X|Y=0}(G) + \pi_1P_{X|Y=1}(G) \\
            &= \pi_0P_{X|Y=0}\Big(G_h \times \{0\} \cup G_h^c \times \{1\}\Big)
                + \pi_1P_{X|Y=1}\Big(G_h \times \{0\} \cup G_h^c \times \{1\}\Big) \\
            &= \pi_0P_{X|Y=0}(G_h) + \pi_1P_{X|Y=1}(G_h^c) \\
            &= \pi_0P_{X|Y=0}(h(X)=1) + \pi_1P(X|Y=1)(h(X)=0) \\
            &= P(h(X) \ne Y) \\
            &= R(h)
    \end{align*}

    \noindent Let $Q=P_{XY}$. We also have:
    \begin{align*}
        \widehat{Q}(G) = \frac{1}{n}\sum_{i=1}^n \1{(X_i, Y_i) \in G_h} = \frac{1}{n}\sum_{i=1}^n \1{h(X_1)\ne Y_i} = \widehat{R_n}(h)
    \end{align*}


    \noindent From the above, we have:
    \begin{align*}
        P\biggRound{
            \sup_{h\in\mathcal{H}} \bigAbs{ \widehat{R_n}(h) - R(h) } \ge \epsilon
        }
        &= P\biggRound{
            \sup_{G\in\mathcal{G}} \bigAbs{ \widehat{Q}(G) - Q(G) } \ge \epsilon
        } \\
        &\le 8S_\mathcal{G}(n)e^{-n\epsilon^2/32} \\
        &=   8S_\mathcal{H}(n)e^{-n\epsilon^2/32} \\
    \end{align*}
\end{solution*}

\begin{exercise}{}{exercise_5.3}
    Let $\G_1$ and $\G_2$ denote two classes of sets:
    \begin{itemize}
        \item ${\bf (a)}$ $\G_1 \cap \G_2=\bigCurl{G_1\cap G_2: G_1\in\G_1, G_2 \in\G_2}$.
        \item ${\bf (b)}$ $\G_1 \cup \G_2=\bigCurl{G_1\cup G_2: G_1\in\G_1, G_2 \in\G_2}$.
    \end{itemize}

    Show that:
\end{exercise}

