\newpage
\section{Kernels and Hilbert Spaces}
\subsection{Pre-Hilbert Spaces}

\begin{definition}[Pre-Hilbert Spaces]
    A \underline{real} inner product space (IPS) is a pair $\bigRound{V, \inner{.,.}}$ where $V$ is a \underline{real} vector space equipped with and inner product $\inner{.,.}: V\times V \to \R$ that satisfies:
    \begin{itemize}
        \item \textbf{Positive semi-definiteness} : $\inner{u,u}\ge0, \ \forall u\in V$ and $\inner{u,u}=0\iff u=0$.
        \item \textbf{Symmetry} : $\inner{u,v} = \inner{v, u}, \ \forall u,v\in V$.
        \item \textbf{Linearity} : $\inner{a_1u_1 + a_2u_2, v} = a_1\inner{u_1, v} + a_2\inner{u_2, v}, \ \forall u_1, u_2, v \in V \text{ and } a_1, a_2 \in \R$.
    \end{itemize}

    \noindent We can also call $\bigRound{V, \inner{.,.}}$ a \textbf{Pre-Hilbert Space}.
\end{definition}

\begin{proposition}{Normed induced by Pre-Hilbert Spaces}{norm_induced_by_pre_hilbert_space}
    If $V$ is a Pre-Hilbert space with an inner product $\inner{.,.}$. Then,
    \begin{align*}
        \|u\| = \sqrt{\inner{u,u}}, \ \forall u \in V
    \end{align*}

    \noindent Is a norm on $V$.
\end{proposition}

\begin{proof*}[Proposition \ref{prop:norm_induced_by_pre_hilbert_space}]
    To prove that $\|.\|$ is a norm on $V$, we have to prove that it satisfies the following properties: absolute homogeneity, positive semi-definiteness and triangle inequality. 

    \begin{itemize}
        \item \textbf{Absolute homogeneity} : Let $\alpha\in\R$ and $u\in V$, we have
        \begin{align*}
            \|\alpha u \| &= \inner{\alpha u, \alpha u}^{\frac{1}{2}} \\
            &= \bigRound{\alpha^2 \inner{u, u}}^{\frac{1}{2}} \ \ \ \text{(Linearity of inner product)} \\
            &= |\alpha| \cdot \inner{u, u}^{\frac{1}{2}} \\
            &= |\alpha| \cdot \|u\|
        \end{align*}

        \item \textbf{Positive semi-definiteness} : This property is inferred directly from the positive semi-definiteness of inner product.
        \item \textbf{Triangle inequality} : Let $u, v\in V$, we have
        \begin{align*}
            \|u + v\| &= \inner{u+v, u+v}^{\frac{1}{2}} \\
            &= \bigRound{\inner{u, u} + 2\inner{u, v} + \inner{v, v}}^{\frac{1}{2}} \\
            &= \bigRound{\|u\|^2 + 2\inner{u, v} + \|v\|^2}^{\frac{1}{2}} \\
            &\le \bigRound{\|u\|^2 + 2\|u\|\cdot\|v\| + \|v\|^2}^{\frac{1}{2}} \ \ \ \text{(Cauchy-Schwarz Inequality)} \\
            &= \|u\| + \|v\|
        \end{align*}
    \end{itemize}

    \noindent Note that the proof for triangle inequality makes use of the Cauchy-Schwarz inequality, which we will prove shortly. 
\end{proof*}

\begin{proposition}{Cauchy-Schwarz Inequality}{cauchy_schwarz_inequality}
    Let $\bigRound{V, \inner{.,.}}$ be a Pre-Hilbert space, we have:
    \begin{align*}
        \Big|\inner{u, v}\Big| \le \|u\|\cdot\|v\|, \ \forall u, v \in V
    \end{align*}
\end{proposition}

\begin{proof*}[Proposition \ref{prop:cauchy_schwarz_inequality}]
    Consider the following matrix:
    \begin{align*}
        G(u, v) = \begin{pmatrix}
            \inner{u, u} & \inner{u, v} 
            \\ \\
            \inner{v, u} & \inner{v, v}
        \end{pmatrix}
        =
        \begin{pmatrix}
            \|u\|^2 & \inner{u, v} 
            \\ \\
            \inner{v, u} & \|v\|^2
        \end{pmatrix}
    \end{align*}

    \noindent Notice that all elements of $G(u, v)$ are non-negative. Hence, $G(u, v)$ is positive semi-definite and we have:
    \begin{align*}
        \det\bigRound{G(u, v)} \ge 0 &\implies \|u\|^2 \cdot \|v\|^2 - \inner{u, v} \cdot \inner{v, u} \ge 0 \\
        &\implies \|u\|^2 \cdot \|v\|^2 \ge \inner{u, v} \cdot \inner{v, u} \\
        &\implies \|u\| \cdot \|v\| \ge \Big| \inner{u, v} \Big|
    \end{align*}
\end{proof*}



\subsection{Hilbert Spaces}
In this section we will discuss what a metric space is and the criteria for a Pre-Hilbert space to be a Hilbert space.

\begin{definition}[Cauchy Sequence]
    Let $M$ be a set and $\{x_n\}_{n=1}^\infty$ be a sequence in $M$. We say that $\{x_n\}_{n=1}^\infty$ is a \textbf{Cauchy sequence} if the elements get closer and closer as $n\to\infty$. Formally, for all $\epsilon > 0$, we have:
    \begin{align*}
        \exists N \in \mathbb{N} \text{ such that: } \forall i, j \ge N, \ d(x_i, x_j) < \epsilon 
    \end{align*}

    \noindent Where $d: M \times M \to \R$ is a \underline{metric}, which is explained in the definition below.
\end{definition}

\begin{definition}[Metric Space]
    A metric space is a pair $\bigRound{M, d}$ where $M$ is a set and $d:M\times M\to \R$ satisfies:
    \begin{itemize}
        \item \textbf{Positive semi-definiteness} : $d(x, y) \ge 0, \ \forall x, y \in M$ and $d(x, y) = 0 \iff x = y$.
        \item \textbf{Symmetry} : $d(x, y) = d(y, x), \ \forall x, y \in M$.
        \item \textbf{Triangle inequality} : $d(x, y) \le d(x, z) + d(y, z), \ \forall x, y, z \in M$.
    \end{itemize}

    \noindent We say that $\bigRound{M, d}$ is a \textbf{complete metric space} if and only if \textbf{every Cauchy sequence converges to an element in $M$}.
\end{definition}

\begin{definition}[Hilbert Space]
    A Pre-Hilbert Space $\bigRound{V, \inner{.,.}}$ is a \textbf{Hilbert Space} if $\bigRound{V, d_{\|.\|}}$ is a complete metric space where $d_{\|.\|}$ is the metric induced by the inner product:
    \begin{align*}
        d_{\|.\|}(x, y) = \|x - y\|, \ x, y \in V
    \end{align*}

    \noindent Where $\|u\| = \sqrt{\inner{u, u}}$ for $u\in V$.
\end{definition}

\subsection{Important theorems in Hilbert Spaces}
In this section, we will discuss two important theorems, namely the \textbf{Hilbert Projection Theorem} and the \textbf{Riesz Representation Theorem}.

\subsubsection{Projection Theorem}
The \textbf{Hilbert Projection Theorem} talks about the condition when any vector in a Hilbert Space will have a unique projection onto a subspace. Before proving the theorem, we look at the Polarization Identity.

\noindent \textbf{Note} : From here, we denote Hilbert Spaces as $H$.

\begin{lemma}{Polarization Identity}{polarization_identity}
    Let $H$ be a Hilbert Space. For all $f,g\in H$, we have:
    \begin{align*}
        \|f + g\|^2 + \|f - g\|^2 = 2\bigRound{ \|f\|^2 + \|g\|^2 }
    \end{align*}
\end{lemma}

\begin{proof*}[Lemma \ref{lem:polarization_identity}]
    We have:
    \begin{align*}
        \|f+g\|^2 &= \inner{f + g, f + g} \\
            &= \|f\|^2 + 2 \inner{f, g} + \|g\|^2 \ \ \ (1) \\
        \|f-g\|^2 &= \inner{f - g, f - g} \\
            &= \|f\|^2 - 2 \inner{f, g} + \|g\|^2 \ \ \ (2) \\
    \end{align*}

    \noindent Combining $(1)$ and $(2)$ yields the Polarization Identity.
\end{proof*}

\begin{theorem}{Hilbert Projection Theorem}{hilbert_projection_theorem}
    Let $H$ be a Hilbert Space and $V\subseteq H$ be a \underline{closed subspace}. For any $f\in H$, there exists a \underline{unique} $v \in V$ such that:
    \begin{align*}
        v = \arg\min_{v^* \in V} \| f - v^*\|
    \end{align*}

    In other words, \textbf{there exists a unique minimizer inside the subspace if and only if the subspace is closed}.
\end{theorem}

\begin{proof*}[Theorem \ref{thm:hilbert_projection_theorem}]
    In this proof, without loss of generality, we work with the minimizer for the norm squared rather than the norm.

    \begin{subproof}{Claim : $V$ is closed $\implies$ unique minimizer exists}
        Suppose that $V$ is a closed subspace of $H$. Since $H$ is a complete metric space, $V$ is also complete. Now let:
        \begin{align*}
            \alpha^2 = \inf_{v^* \in V } \| f - v^*\|^2
        \end{align*}

        \noindent Since for all $v\in V$, we have $\|f - v\|^2 \ge \alpha^2$. Choose a sequence $\{v_n\}_{n=1}^\infty \subset V$ such that:
        \begin{align*}
            0 \le \|f - v_n \|^2 - \alpha^2 \le \frac{1}{n} \ \ \ (1)
        \end{align*}

        \noindent For any $m, n \in \mathbb{N}$, we have:
        \begin{align*}
            \|f - v_n\|^2 + \|f - v_m\|^2 - 2\alpha^2 \le \frac{1}{m} + \frac{1}{n}
        \end{align*}

        \noindent By Polarization Identity \ref{lem:polarization_identity}, we have $\|f-v_n\|^2 + \|f-v_m\|^2 = \frac{1}{2}\bigRound{\|v_n-v_m\|^2 + \|2f - (v_n+v_m)\|^2}$. Hence, we have:
        \begin{align*}
            \|v_n-v_m\|^2 + \|2f - (v_n+v_m)\|^2 - 4\alpha^2 &\le \frac{2}{m} + \frac{2}{n} \\
            \implies \|v_n-v_m\|^2 + 4\biggRound{
                \Bigg\| f - \frac{v_n + v_m}{2} \Bigg\|^2 - \alpha^2
            } &\le \frac{2}{m} + \frac{2}{n} \ \ \ (2)
        \end{align*}

        \noindent Since $\frac{v_n + v_m}{2}\in V$, we have:
        \begin{align*}
            \Bigg\| f - \frac{v_n + v_m}{2} \Bigg\|^2 \ge \alpha \implies \|v_n-v_m\|^2 + 4\biggRound{
                \Bigg\| f - \frac{v_n + v_m}{2} \Bigg\|^2 - \alpha^2
            } \ge \|v_n-v_m\|^2 \ \ \ (3)
        \end{align*}

        \noindent From $(2)$ and $(3)$, we have:
        \begin{align*}
            \|v_n-v_m\|^2 \le \frac{2}{m} + \frac{2}{n} 
        \end{align*}

        \noindent Taking $m, n \to \infty$, we have $\|v_n-v_m\|^2\to 0$. Therefore, $\{v_n\}_{n=1}^\infty$ is a Cauchy sequence. Since $V$ is a complete metric space, $v_n \to v \in V$. From $(1)$ we have:
        \begin{align*}
            \|f - v\|^2 = \lim_{n\to\infty} \|f - v_n\|^2 = \alpha^2 
        \end{align*}

        \noindent Hence, We have proven that a minimizer indeed exists in $V$. Now, we have to prove that the minimizer is actually unique. Suppose that we have another $\hat v \in V$ such that:
        \begin{align*}
            \hat v = \arg\min_{v^*\in V} \|f - v^*\|^2
        \end{align*}

        \noindent Then, by the Polarization Identity \ref{lem:polarization_identity}, we have:
        \begin{align*}
            2\alpha^2 &= \|f - v\|^2 + \|f - \hat v\|^2 \\
                &= \frac{1}{2}\bigRound{\|v - \hat v\|^2 + \| 2f - (v + \hat v) \|^2} \ \ \ \text{(Polarization Identity)} \\
                &= \frac{1}{2} \|v - \hat v \|^2 + 2\Bigg\| f - \frac{v + \hat v}{2} \Bigg\|^2 \\
                &\ge \frac{1}{2} \|v - \hat v \|^2 + 2\alpha^2 \ \ \ \Bigg( \text{Since } \frac{v+\hat v}{2} \in V \Bigg)
        \end{align*}

        \noindent From the above, we have:
        \begin{align*}
            0 \ge \frac{1}{2} \|v - \hat v \|^2 \ge 0 \implies \|v - \hat v \|^2 = 0
        \end{align*}

        \noindent Therefor, $v = \hat v$ and the minimizer is unique.
    \end{subproof}

    \begin{subproof}{\newline Claim : unique minimizer exists $\implies$ $V$ is closed}
        Suppose that $V$ is not closed, meaning it does not contain all its limit points. Hence, there exists a sequence $\{f_n\}_{n=1}^\infty \subset V$ such that $f_n \to f \in H \setminus V$. By the assumption, there exists $v\in V$ such that:
        \begin{align*}
            \|f - v\|^2 = \inf_{v^*\in V}\|f - v^*\|^2
        \end{align*}

        \noindent But then, we have:
        \begin{align*}
            0 \le \|f - v\|^2 \le \|f - f_n\|^2
        \end{align*}

        \noindent Therefore,
        \begin{align*}
            0 \le \|f - v\|^2 \le \lim_{n\to\infty} \|f - f_n\|^2 = 0 \implies f = v
        \end{align*}

        \noindent This means that $f \in V \implies$ contradiction. Hence, we can conclude that $V$ is a closed subspace of $H$.
    \end{subproof}
\end{proof*}

\begin{corollary}{Hilbert Spaces as direct sum}{hilbert_space_direct_sum}
    Let $H$ be a Hilbert Space and $V\subseteq H$ is a closed subspace. Then, we can write:
    \begin{align*}
        H = V \oplus V^\perp
    \end{align*}

    \noindent In other words, we can define:
    \begin{align*}
        H = \bigCurl{
            x + y \Big| x\in V, y \in V^\perp 
        }
    \end{align*}
\end{corollary}

\begin{proof*}[Corollary \ref{coro:hilbert_space_direct_sum}]
    
\end{proof*}


\subsubsection{Representation Theorem}
