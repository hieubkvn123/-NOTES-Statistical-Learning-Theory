\newpage\section{Rademacher Complexity}
\subsection{Bounded Difference Inequality}
In the following section, we will discuss another concentration inequality that bounds the difference between functions of random sample and their mean given that the functions satisfy the \textbf{bounded difference property}.

\begin{definition}[Bounded difference property]
    Given a real-valued function $\phi:\X^n\to\R$. We say that $\phi$ satisfies the \textbf{bounded difference property} if $\exists c_1, \dots, c_n\in\X$ such that $\forall 1 \le i \le n$:
    \begin{align*}
        \sup_{\{x_1, \dots, x_n\}\subset\X, x_i' \in \X}
        \biggAbs{
            \phi(x_1, \dots, x_i, \dots, x_n) - \phi(x_1, \dots, x_i', \dots, x_n)
        } \le c_i
    \end{align*}

    \noindent That is, substituting the value at the $i^{th}$ coordinate $x_i$ changes the value of $\phi$ by at most $c_i$.
\end{definition}

\begin{theorem}{Bounded Difference (McDiarmid's) Inequality}{bounded_diff_inequality}
    Let $X_1, \dots, X_n$ be independent random variables (not necessarily identically distributed) and $\phi:\X^n \to \R$ be a function satisfying the bounded difference property:
    \begin{align*}
        \sup_{\{x_1, \dots, x_n\}\subset\X, x_i' \in \X}
        \biggAbs{
            \phi(x_1, \dots, x_i, \dots, x_n) - \phi(x_1, \dots, x_i', \dots, x_n)
        } \le c_i, \ \forall 1\le i \le n
    \end{align*}

    \noindent Then, we have:
    \begin{align*}
        P\biggRound{
            \bigAbs{
                \phi(X_1, \dots, X_n) - \mathbb{E}\bigSquare{ \phi(X_1, \dots, X_n) }
            } \ge t
        } \le 2\exp\biggRound{
            -\frac{2t^2}{\sum_{i=1}^n c_i^2}
        }, \ \forall t > 0
    \end{align*}
\end{theorem}

\noindent\textbf{Remark} : Assume that $X_i\in[a_i, b_i]$ and $\phi(X_1, \dots, X_n)=\sum_{i=1}^n X_i$. Then the bounded difference inequality recovers the Hoeffding's inequality \ref{thm:hoeffding_inequality}.

\begin{proof*}[Theorem \ref{thm:bounded_diff_inequality}]
    Define the following random variable:
    \begin{align*}
        V_i = \mathbb{E}\Big[\phi\Big|X_1, \dots, X_i\Big] - \mathbb{E}\Big[\phi\Big|X_1, \dots, X_{i-1}\Big]
    \end{align*}

    \noindent Denote $\phi(X_1, \dots, X_n)=\phi$ and $\mathbb{E}[\phi(X_1, \dots, X_n)] = \mu_\phi$ for brevity, we have:
    \begin{align*}
        \phi - \mu_\phi = \sum_{i=1}^n V_i
    \end{align*}

    \noindent Using the Chernoff's bounding method, we have:
    \begin{align*}
        P(\phi - \mu_\phi \ge t) 
            &\le \inf_{s>0}e^{-st}M_{\phi-\mu_\pi}(s) \\
            &= \inf_{s>0}e^{-st}\mathbb{E}\biggSquare{
                \exp\biggRound{
                    s\sum_{i=1}^n V_i
                }
            }
    \end{align*}

    \begin{subproof}{\newline Claim 1 : For all $1\le i\le n$, $a_i\le V_i \le b_i$ and $b_i-a_i\le c_i$.}
        Define the infimum and supremum of $V_i$ as followed:

        \begin{align*}
            U_i &= \sup_{x\in\X}\biggCurl{
                \E\bigSquare{\phi \Big| X_1, \dots, X_i=x} - \E\bigSquare{\phi\Big|X_1, \dots, X_{i-1}}
            } \\
            L_i &= \inf_{x\in\X}\biggCurl{
                \E\bigSquare{\phi \Big| X_1, \dots, X_i=x} - \E\bigSquare{\phi\Big|X_1, \dots, X_{i-1}}
            }
        \end{align*}

        \noindent Clearly, $U_i \ge V_i \ge L_i$. We have:
        \begin{align*}
            U_i - L_i
                &= \sup_{x\in\X}\E\bigSquare{\phi\Big|X_1, \dots, X_{i-1}, X_i=x} - \inf_{x\in\X}\E\bigSquare{\phi\Big|X_1, \dots, X_{i-1}, X_i=x} \\
                &= \sup_{x\in\X}\int\phi(X_1, \dots, X_{i-1}, x, x_{i+1}, \dots, x_n)dP(x_{i+1}, \dots, x_n|X_1, \dots, X_{i-1}, x) \\
                & \ \ \ -\inf_{x\in\X}\int\phi(X_1, \dots, X_{i-1}, x, x_{i+1}, \dots, x_n)dP(x_{i+1}, \dots, x_n|X_1, \dots, X_{i-1}, x) \\
                &= \sup_{x\in\X}\int\phi(X_1, \dots, X_{i-1}, x, x_{i+1}, \dots, x_n)dP(x_{i+1}, \dots, x_n) \\
                & \ \ \ -\inf_{x\in\X}\int\phi(X_1, \dots, X_{i-1}, x, x_{i+1}, \dots, x_n)dP(x_{i+1}, \dots, x_n) \\
                &= \sup_{x, y\in \X} \int\bigSquare{
                    \phi(X_1, \dots, X_{i-1}, x, x_{i+1}, \dots, x_n) - \phi(X_1, \dots, X_{i-1}, y, x_{i+1}, \dots, x_n)
                }dP(x_{i+1}, \dots, x_n) \\
                &\le c_i \int dP(x_{i+1}, \dots, x_n) = c_i
        \end{align*}
    \end{subproof}

    \begin{subproof}{\newline Claim 2 : $\E\bigSquare{ V_i\Big| X_1, \dots, X_{i-1} } = 0, \ \forall 1 \le i \le n$.}
        We have:
        \begin{align*}
            \E\bigSquare{ V_i\Big| X_1, \dots, X_{i-1} } 
                &= \E\biggSquare{ \E\bigSquare{\phi\Big|X_1, \dots, X_i} \Bigg| X_1, \dots, X_{i-1} }
                - \E\bigSquare{ \E\bigSquare{\phi\Big|X_1, \dots, X_{i-1}} \Big| X_1, \dots, X_{i-1} } \\
                &= \E\biggSquare{ \E\bigSquare{\phi\Big|X_1, \dots, X_i} \Bigg| X_1, \dots, X_{i-1} }
                - \E\bigSquare{\phi\Big|X_1, \dots, X_{i-1}}
        \end{align*}

        \noindent By the tower property, we have:
        \begin{align*}
            \E\biggSquare{ \E\bigSquare{\phi\Big|X_1, \dots, X_i} \Bigg| X_1, \dots, X_{i-1} } = \E\bigSquare{\phi\Big|X_1, \dots, X_{i-1}}
        \end{align*}

        \noindent Hence, 
        \begin{align*}
            \E\bigSquare{ V_i\Big| X_1, \dots, X_{i-1} } &= \E\bigSquare{\phi\Big|X_1, \dots, X_{i-1}} - \E\bigSquare{\phi\Big|X_1, \dots, X_{i-1}} \\ &=0
        \end{align*}
    \end{subproof}

    \noindent From the above claims, we can make use of the Hoeffding's lemma \ref{lem:hoeffding_lemma} to bound the moment generating functions. We have:
    \begin{align*}
        P(\phi - \mu_\phi \ge t) 
        &= \inf_{s>0}e^{-st}\mathbb{E}\biggSquare{
            \exp\biggRound{
                s\sum_{i=1}^n V_i
            }
        } \\
        &= \inf_{s>0}e^{-st}\E_{X_1, \dots, X_{n-1}}\E_{X_n|X_1, \dots, X_{n-1}}\biggSquare{
            \exp\biggRound{
                s\sum_{i=1}^n V_i
            }
        } \\
        &= \inf_{s>0}e^{-st}\E_{X_n|X_1, \dots, X_{n-1}}\bigSquare{e^{sV_n}}\E_{X_1, \dots, X_{n-1}}\biggSquare{
            \exp\biggRound{
                s\sum_{i=1}^{n-1} V_i
            }
        }\\
        &\le \inf_{s>0}\exp\biggRound{
            -st + \frac{s^2c_n^2}{8}
        }\E_{X_1, \dots, X_{n-1}}\biggSquare{
            \exp\biggRound{
                s\sum_{i=1}^{n-1} V_i
            }
        }  \ \ \ \text{(Lemma \ref{lem:hoeffding_lemma})} \\
        &\vdots \\
        &\le \inf_{s>0}\exp\biggRound{
            -st + s^2 \sum_{i=1}^n \frac{c_i^2}{8}
        }
    \end{align*}

    \noindent Substituting $s=\frac{4t}{\sum_{i=1}^n c_i^2}$ to minimize the upperbound (just like the proof for Hoeffding's inequality \ref{thm:hoeffding_inequality}). We have:
    \begin{align*}
        P(\phi - \mu_\phi \ge t) \le \exp\biggRound{
            -\frac{2t^2}{\sum_{i=1}^n c_i^2}
        }
    \end{align*}
\end{proof*}

\subsection{Rademacher Complexity}
\textbf{Overview} : Rademacher Complexity is a measure for the richness of a class of real-valued functions. In this sense, it is similar to VC dimension. However, unlike VC dimension, the Rademacher Complexity is not restricted to binary functions.

\begin{definition}[Empirical Rademacher Complexity]
    Let $\G\subseteq[a,b]^{\mathcal{Z}}$ be a set of functions $\mathcal{Z} \to [a, b]$ where $a,b\in\R, a < b$. Let $Z_1, \dots, Z_n$ be an independently identically distributed random sample on $\mathcal{Z}$ following some distribution $P$. Denote $S=(Z_1, \dots, Z_n)$, we define the \textbf{Empirical Rademacher Complexity} as:
    \begin{align*}
        \ERC_S(\G) = \E_\sigma\biggSquare{
            \sup_{g\in\G} \frac{1}{n}\sum_{i=1}^n \sigma_i g(Z_i)
        }
    \end{align*}

    \noindent Where $\sigma=(\sigma_1, \dots, \sigma_n)^T$, $\sigma_i\sim Uniform(-1, 1)$ are known as \textbf{Rademacher random variables}. Note that $\ERC_S(\G)$ is random due to randomness in $S$.
\end{definition}

\begin{definition}[Rademacher Complexity]
    The \textbf{Rademacher Complexity} of a function class $\G$ is defined as:
    \begin{align*}
        \RC_n(\G) = \E_S\bigSquare{ \ERC_S(\G) }
    \end{align*}
\end{definition}

\begin{theorem}{One-sided Rademacher Complexity bound}{one_sided_rademacher_bound}
    Let $Z$ be a random variable and $S=(Z_1, \dots, Z_n)$ be an independently identically distributed sample over $\mathcal{Z}$. Consider a class of functions $\G\subseteq [a,b]^\mathcal{Z}$. $\forall \delta > 0, g\in\G$, with at least probability $1-\delta$ with respect to the draw of sample $S$, we have:
    \begin{align*}
        {\bf (i)} \ \ \sup_{g\in\G}\biggCurl{\E\bigSquare{g(Z)} - \frac{1}{n} \sum_{i=1}^n g(Z_i)} &\le 2\RC_n(\G) + (b-a)\sqrt{\frac{\log1/\delta}{2n}} \\
        {\bf (ii)} \  \sup_{g\in\G}\biggCurl{\E\bigSquare{g(Z)} - \frac{1}{n} \sum_{i=1}^n g(Z_i)} &\le 2\ERC_S(\G) + 3(b-a)\sqrt{\frac{\log2/\delta}{2n}}
    \end{align*}
\end{theorem}

\begin{proof*}[Theorem \ref{thm:one_sided_rademacher_bound}]
    For notation brevity, define:
    \begin{align*}
        \widehat{\E_S}[g] &= \frac{1}{n}\sum_{i=1}^n g(Z_i); \ \E\bigSquare{g(Z)} = \E[g] \\
    \end{align*}
    
    \begin{subproof}{\newline ${\bf (i)} \ \ \E[g] - \widehat{\E_S}[g] \le 2\RC_n(\G) + (b-a)\sqrt{\frac{\log1/\delta}{2n}}$}
        Define the following function:
        \begin{align*}
            \phi(S) = \sup_{g\in\G} \bigCurl{
                \E[g] - \widehat{\E_S}[g]
            }
        \end{align*}

        \noindent First, we check that $\phi$ has the bounded difference property. Define $S'_i=(Z_1, \dots, Z_i', \dots, Z_n)$, we have:
        \begin{align*}
            \bigAbs{\phi(S) - \phi(S'_i)} 
                &= \biggAbs{
                    \sup_{g\in\G} \bigCurl{\E[g] - \widehat{E_S}[g]} - \sup_{g\in\G}\bigCurl{\E[g] - \widehat{\E_{S_i'}}[g]}
                } \\
                &\le \biggAbs{
                    \sup_{g\in\G} \bigCurl{ \widehat{\E_{S_i'}}[g] - \widehat{E_S}[g] }
                } \ \ \ \Big(\sup A - \sup B \le \sup \bigCurl{A - B} \Big) \\
                &= \biggAbs{
                    \frac{1}{n} \sup_{g\in\G} \bigRound{ g(Z_i') - g(Z_i) }
                } \\
                &\le \frac{b-a}{n}
        \end{align*}

        \noindent By the Bounded Difference Inequality, we have:
        \begin{align*}
            P\biggRound{
                \phi(S) - \E[\phi(S)] \le t
            } &\ge 1 - \exp\biggRound{
                -\frac{2t^2}{\sum_{i=1}^n (b - a)^2/n^2}
            }, \ \ \ t\ge 0 \\
            &= 1 - \exp\biggRound{
                -\frac{2nt^2}{(b - a)^2}
            }
        \end{align*}

        \noindent Now, let:
        \begin{align*}
            \delta = \exp\biggRound{
                -\frac{2nt^2}{(b - a)^2}
            } \implies t = (b-a)\sqrt{
                \frac{\log 1/\delta}{2n}
            }
        \end{align*}

        \noindent Hence, for all $\delta>0$, with probability of at least $1-\delta$, we have:
        \begin{align*}
            \phi(S) \le \E[\phi(S)] + (b-a)\sqrt{
                \frac{\log 1/\delta}{2n}
            } \ \ \ (1)
        \end{align*}
    \end{subproof}

    \noindent Now, to establish ${\bf (i)}$, we have to show that $\E[\phi(S)] \le 2\RC_n(\G)$. Let $S'=(Z_1', \dots, Z_n')$. We have:
    \begin{align*}
        \E[\phi(S)] 
            &= \E_S\biggSquare{ \sup_{g\in\G} \bigCurl{ \E[g] - \widehat{E_S}[g] } } \\
            &= \E_S\biggSquare{
                \sup_{g\in\G} \bigCurl{
                    E_{S'} \bigSquare{ \widehat{E_{S'}}[g] - \widehat{E_S}[g]}
                }
            } \ \ \ \Big(\E[g] = \E_{S'}\widehat{E_{S'}}[g]\Big) \\
            &\le \E_{S, S'} \biggSquare{
                \sup_{g\in\G} \bigCurl{
                    \widehat{E_{S'}}[g] - \widehat{E_S}[g]
                }
            } \ \ \ \Big( \text{Jensen's Inequality : } \sup \E \le \E\sup \Big) \\
            &= \E_{S, S'} \biggSquare{
                \sup_{g\in\G} \frac{1}{n}\sum_{i=1}^n \bigRound{
                    g(Z_i') - g(Z_i)
                }
            } \\
            &= \E_{\sigma, S, S'} \biggSquare{
                \sup_{g\in\G} \frac{1}{n}\sum_{i=1}^n \sigma_i\bigRound{
                    g(Z_i') - g(Z_i)
                }
            } \ \ \ (Z_i, Z_i' \text{ are i.i.d, }\sigma_i \text{ are symmetric}) \\
            &\le E_{\sigma, S'} \biggSquare{
                \sup_{g\in\G} \frac{1}{n}\sum_{i=1}^n \sigma_ig(Z_i')
            } + E_{\sigma, S} \biggSquare{
                \sup_{g\in\G} \frac{1}{n}\sum_{i=1}^n (-\sigma_i)g(Z_i)
            } \ \ \ (\sup(f_1 + f_2) \le \sup f_1 + \sup f_2) \\
            &=  E_{\sigma, S'} \biggSquare{
                \sup_{g\in\G} \frac{1}{n}\sum_{i=1}^n \sigma_ig(Z_i')
            } + E_{\sigma, S} \biggSquare{
                \sup_{g\in\G} \frac{1}{n}\sum_{i=1}^n \sigma_ig(Z_i)
            } \ \ \ (\text{Rademacher variables are symmetric})  \\
            &= 2\RC_n(\G) \ \ \ (2)
    \end{align*}

    \noindent From $(1)$ and $(2)$, we have:
    \begin{align*}
        \phi(S) \le 2\RC_n(\G) + (b-a)\sqrt{
            \frac{\log 1/\delta}{2n}
        }
    \end{align*}

    \begin{subproof}{\newline ${\bf (ii)} \ \ \E[g] - \widehat{\E_S}[g] \le 2\ERC_n(\G) + 3(b-a)\sqrt{\frac{\log2/\delta}{2n}}$}
        We will first verify that the Empirical Rademacher Complexity satisfies the bounded difference property. Let $S=(X_1, \dots, X_n)$ and $S_i'=(Y_1, \dots, Y_n)$ such that $Y_j=X_j, \forall j \ne i, 1 \le i \le n$. We have:
        \begin{align*}
            \bigAbs{ \ERC_S(\G) - \ERC_{S_i'}(\G) } 
                &\le \biggAbs{
                    \E_\sigma\biggSquare{
                        \sup_{g\in\G} \frac{1}{n} \biggRound{
                            \sum_{j=1}^n \sigma_j g(X_j) - \sum_{j=1}^n \sigma_j g(Y_j)
                        }
                    }
                } \\
                &= \biggAbs{
                    \E_\sigma\biggSquare{
                        \sup_{g\in\G} \frac{1}{n} \sum_{j=1}^n \sigma_j (g(X_j) - g(Y_j))
                    }
                } \\
                &= \biggAbs{
                    \E_\sigma\biggSquare{
                        \sup_{g\in\G} \frac{1}{n} \sum_{j=1}^n (g(X_j) - g(Y_j))
                    }
                } \ \ \ (\sigma_i \text{ are symmetric}) \\
                &= \biggAbs{
                    \E_\sigma\biggSquare{
                        \sup_{g\in\G} \frac{1}{n} \bigRound{
                            g(X_i) - g(Y_i)
                        }
                    }
                } \\
                &\le \frac{b-a}{n}
        \end{align*}

        \noindent Therefore, by the bounded difference inequality, with at least $1-\delta/2$ probability, we have:
        \begin{align*}
            \ERC_S(\G) - \E_S\bigSquare{\ERC_S(\G)} &\ge (a-b)\sqrt{\frac{\log 2/\delta}{2n}} \\
            \implies \ERC_S(\G) - \RC_n(\G) &\ge (a-b)\sqrt{\frac{\log 2/\delta}{2n}} \\
            \implies \RC_n(\G)  &\le \ERC_S(\G) + (b-a)\sqrt{\frac{\log 2/\delta}{2n}}
        \end{align*}

        \noindent From ${\bf (i)}$ we also have, with a probability of at least $1-\delta/2$, we have:
        \begin{align*}
            \phi(S) \le 2\RC_n(\G) + (b-a)\sqrt{
                \frac{\log 2/\delta}{2n}
            }
        \end{align*}

        \noindent Now denote the following events:
        \begin{align*}
            A &:= \biggCurl{
                \RC_n(\G) \le \ERC_S(\G) + (b-a)\sqrt{ \frac{\log 2/\delta}{2n} }
            } \\
            B &:= \biggCurl{
                \phi(S) \le 2 \RC_n(\G) + (b-a)\sqrt{ \frac{\log 2/\delta}{2n} }
            }
        \end{align*}

        \noindent We have:
        \begin{align*}
            P\biggRound{\biggCurl{
                \phi(S) \le 2\ERC_S(\G) + 3(b-a)\sqrt{
                    \frac{\log 2/\delta}{2n}
                }
            }} &\ge P(A\cap B) \\
            &= 1 - P(\overline{A} \cup \overline{B}) \\
            &\ge 1 - \bigRound{P(\overline{A}) + P(\overline{B})} \\
            &= 1 - (\delta/2 + \delta/2) = 1 - \delta
        \end{align*}
    \end{subproof}
\end{proof*}

\begin{theorem}{Two-sided Rademacher Complexity bound}{two_sided_rademacher_bound}
    Consider a set of classifiers $\G\subseteq[a,b]^\mathcal{Z}$. Then, $\forall \delta>0$, with probability of at least $1-\delta$ with respect to the draw of sample $S$, we have:
    \begin{align*}
        \sup_{g\in\G}\biggAbs{
            \E\bigSquare{ g(Z) } - \frac{1}{n}\sum_{i=1}^n g(Z_i)
        } &\le 2 \RC_n(\G) + (b-a)\sqrt{
            \frac{\log 2/\delta}{2n}
        } \\
        \sup_{g\in\G}\biggAbs{
            \E\bigSquare{ g(Z) } - \frac{1}{n}\sum_{i=1}^n g(Z_i)
        } &\le 2 \ERC_n(\G) + 3(b-a)\sqrt{
            \frac{\log 4/\delta}{2n}
        }
    \end{align*}
\end{theorem}

\begin{proof*}[Theorem \ref{thm:two_sided_rademacher_bound}]
    The proof of this theorem is included in exercise \ref{ex:exercise_6.4}
\end{proof*}


\subsection{Bounds for binary classification}
\textbf{Overview} : Given the following, 
\begin{itemize}
    \item $\X$ is a feature space and $\Y=\{-1, 1\}$ be a label space.
    \item $\Hf\subset\{-1, 1\}^{\X}$ : A set of binary classifiers.
    \item $\G=\bigCurl{ g_h : \X\times\Y \to \{0,1\} \Big| g_h(x, y) = \1{h(x)\ne y}, h\in\Hf }$ : Every function $g_h\in\G$ is the risk function of $h\in\Hf$.
    \item $S=\{(X_i, Y_i)\}_{i=1}^n\subset \X\times \Y$ is a sample dataset.
\end{itemize}

\begin{lemma}{$\ERC_S(\G) = \frac{1}{2}\ERC_S(\Hf)$}{rh_equals_half_rg}
    We have the following equality:
    \begin{align*}
        \ERC_S(\G) = \frac{1}{2}\ERC_S(\Hf)
    \end{align*}
\end{lemma}

\begin{proof*}[Lemma \ref{lem:rh_equals_half_rg}]
    From definition, we have:
    \begin{align*}
        \ERC_S(\G) 
            &= \E_\sigma\biggSquare{
                \sup_{h\in\Hf} \frac{1}{n} \sum_{i=1}^n \sigma_i \1{h(X_i)\ne Y_i} 
            } \\
            &= \E_\sigma\biggSquare{
                \sup_{h\in\Hf} \frac{1}{n} \sum_{i=1}^n \sigma_i \frac{1-Y_ih(X_i)}{2} 
            } \\
            &= \E_\sigma\biggSquare{
                \frac{1}{2n}\sum_{i=1}^n \sigma_i + \frac{1}{2}\sup_{h\in \Hf}\frac{1}{n} \sum_{i=1}^n \sigma_i(-Y_i)h(X_i)
            } \\ 
            &= \frac{1}{2n} \underbrace{\E_\sigma\biggSquare{
                \sum_{i=1}^n \sigma_i
            }}_{=0} + \frac{1}{2}\E_\sigma\biggSquare{
                \sup_{h\in \Hf}\frac{1}{n} \sum_{i=1}^n \sigma_i(-Y_i)h(X_i)
            } \\
            &= \frac{1}{2}\E_\sigma\biggSquare{
                \sup_{h\in \Hf}\frac{1}{n} \sum_{i=1}^n \sigma_i h(X_i)
            } \ \ \ (\sigma_i \text{ and } \sigma_i(-Y_i) \text{ has the same distribution}) \\
            &= \frac{1}{2}\ERC_S(\Hf)
    \end{align*}
\end{proof*}

\begin{corollary}{UDB for binary classification using Rademacher Complexity}{udb_bin_clf_rademacher}
    For all $\delta>0$, with probability of at least $1-\delta$, we have:
    \begin{align*}
        \sup_{h\in\Hf} \bigRound{
            R(h) - \widehat{R_n}(h)
        } &\le \RC_n(\Hf) + \sqrt{
            \frac{\log 1/\delta}{2n}
        } \\
        \sup_{h\in\Hf} \bigRound{
            R(h) - \widehat{R_n}(h)
        } &\le \ERC_S(\Hf) + 3\sqrt{
            \frac{\log 2/\delta}{2n}
        }
    \end{align*}

    \noindent We can also derive a two-sided UDB by replacing $\delta$ with $\delta/2$:
    \begin{align*}
        \sup_{h\in\Hf} \bigAbs{
            R(h) - \widehat{R_n}(h)
        } &\le \RC_n(\Hf) + \sqrt{
            \frac{\log 2/\delta}{2n}
        } \\
        \sup_{h\in\Hf} \bigAbs{
            R(h) - \widehat{R_n}(h)
        } &\le \ERC_S(\Hf) + 3\sqrt{
            \frac{\log 4/\delta}{2n}
        }
    \end{align*}
\end{corollary}

\begin{proof*}[Corollary \ref{coro:udb_bin_clf_rademacher}]
    Define the following function space:
    \begin{align*}
        \G = \bigCurl{
            g_h : \X \times \Y \to \{0,1\} \Big| g_h(x, y) = \1{h(x)\ne y}, h\in\Hf
        }
    \end{align*}

    \noindent By theorem \ref{thm:one_sided_rademacher_bound}, we have:
    \begin{align*}
        \sup_{g_h\in\G} \bigRound{\E[g_h] - \widehat{E_S}[g_h]}
            &= \sup_{h\in\Hf} \bigRound{\E[\1{h(X)\ne Y}] - \widehat{E_S}[\1{h(X)\ne Y}]} \\
            &= \sup_{h\in\Hf} \bigRound{R(h) - \widehat{R_n}(h)} \\
            &\le 2\RC_n(\G) + \sqrt{\frac{\log 1/\delta}{2n}} \ \ \ (\text{Theorem \ref{thm:one_sided_rademacher_bound}}) \\
            &= \RC_n(\Hf) + \sqrt{\frac{\log 1/\delta}{2n}} \ \ \ \biggRound{
                \text{Since } \ERC_S(\G) = \frac{1}{2}\ERC_S(\Hf)
            }
    \end{align*}

    \noindent Similarly, from theorem \ref{thm:one_sided_rademacher_bound}, we can also derive the following inequality:
    \begin{align*}
        \sup_{h\in\Hf} \bigRound{R(h) - \widehat{R_n}(h)} \le \ERC_S(\Hf) + 3\sqrt{\frac{\log 2/\delta}{2n}}
    \end{align*}

    \noindent By replacing $\delta$ with $\delta/2$, following the same arguments, we obtain the two-sided bounds.
\end{proof*}

\subsection{Tighter VC inequalities}
\label{sec:proof_of_vc_inequality}

In the following section, we will prove tighter VC inequalities compared to theorem \ref{thm:vc_theorem_for_clf} with the help of Massart's lemma \ref{thm:massart_lemma}. After that, we will go ahead and look at even tighter versions in the practice exercises.

\begin{theorem}{One-sided VC Inequality}{one_sided_vc_inequality}
    For $0 < \delta < 1$, with probability of at least $1-\delta$, we have:
    \begin{align*}
        \sup_{h\in\Hf} \bigRound{
            R(h) - \widehat{R_n}(h)
        } \le \sqrt{
            \frac{8(\log S_\Hf(n) + \log(1/\delta))}{n}
        }
    \end{align*}

    \noindent Equivalently, for all $\epsilon>0$, we have:
    \begin{align*}
        P\biggRound{
            \sup_{h\in\Hf} \bigRound{
                R(h) - \widehat{R_n}(h)
            } \ge \epsilon
        } \le S_\Hf(n)e^{-n\epsilon^2/8}
    \end{align*}
\end{theorem}

\begin{proof*}[Theorem \ref{thm:one_sided_vc_inequality}]
    Let $\Hf\subset\{-1,1\}^\X$ and $S=(X_1, \dots, X_n)$ be a random sample. Denote the restriction of $\Hf$ to $S$ as:
    \begin{align*}
        \Hf_S = \bigCurl{
            (h(X_1), \dots, h(X_n)) : h \in \Hf
        }
    \end{align*}

    \noindent Hence, for any $u\in\Hf_S$, we have $\|u\|_2 = \sqrt{n}$. By Massart's lemma \ref{thm:massart_lemma}, we have:
    \begin{align*}
        \RC_n(\Hf) &= \E_S\E_\sigma\biggSquare{
            \sup_{u\in \Hf_S} \frac{1}{n}\sum_{i=1}^n \sigma_i u_i
        } \\
        &\le \E_S\biggSquare{
            \frac{\sup_{u\in\Hf} \|u\|_2 \sqrt{2\log |\Hf_S|}}{n}
        } \ \ \ (\text{Massart's lemma \ref{thm:massart_lemma}}) \\
        &=  \E_S\biggSquare{
            \frac{\sqrt{n}\sqrt{2\log |\Hf_S|}}{n}
        } =  \E_S\biggSquare{
            \sqrt{\frac{2\log |\Hf_S|}{n}}
        } \\
        &\le \sqrt{\frac{2\log \E[|\Hf_S|]}{n}} \ \ \ (\text{Jensen's inequality}) \\
        &\le \sqrt{\frac{2\log S_\Hf(n)}{n}} \ \ \ (|\Hf_S| \le S_\Hf(n))
    \end{align*}

    \noindent Combine the above inequality with corollary \ref{coro:udb_bin_clf_rademacher}, with probability of at least $1-\delta$, we have:
    \begin{align*}
        \sup_{h\in\Hf} \bigRound{
            R(h) - \widehat{R_n}(h)
        } &\le \RC_n(\Hf) + \sqrt{
            \frac{\log 1/\delta}{2n}
        } \\
        &\le \sqrt{\frac{2\log S_\Hf(n)}{n}} + \sqrt{\frac{\log 1/\delta}{2n}} \\
        &\le 2\sqrt{
            \frac{2\log S_\Hf(n)}{n} + \frac{\log 1/\delta}{2n}
        } \ \ \ (\sqrt{a} + \sqrt{b} \le 2\sqrt{a+b}) \\
        &= \sqrt{
            \frac{8(\log S_\Hf(n) + (\log 1/\delta)/4)}{n}
        }  \tag{*} \\
        &\le \sqrt{
            \frac{8(\log S_\Hf(n) + \log 1/\delta)}{n}
        }
    \end{align*}

    \noindent Now we set:
    \begin{align*}
        \epsilon = \sqrt{
            \frac{8(\log S_\Hf(n) + \log 1/\delta)}{n}
        } \implies \delta = S_\Hf(n)e^{-n\epsilon^2/8}
    \end{align*}

    \noindent Hence, for all $\epsilon > 0$, we have:
    \begin{align*}
        P\biggRound{
            \sup_{h\in\Hf} \bigRound{
                R(h) - \widehat{R_n}(h)
            } \ge \epsilon
        } \le S_\Hf(n)e^{-n\epsilon^2/8}
    \end{align*}
\end{proof*}

\begin{theorem}{Two-sided VC Inequality}{two_sided_vc_inequality}
    For $0<\delta<1$, with probability of at least $1-\delta$, we have:
    \begin{align*}
        \sup_{h\in\Hf} \bigAbs{
            R(h) - \widehat{R_n}(h)
        } \le \sqrt{
            \frac{8(\ln S_\Hf(n) + \ln(2/\delta))}{n}
        }
    \end{align*}

    \noindent Equivalently, for any $\epsilon>0$, 
    \begin{align*}
        P\biggRound{
            \sup_{h\in\Hf} \bigAbs{
                R(h) - \widehat{R_n}(h)
            }\ge \epsilon
        }\le 2S_\Hf(n)e^{-n\epsilon^2/8}
    \end{align*}
\end{theorem}

\begin{proof*}[Theorem \ref{thm:two_sided_vc_inequality}]
    The proof of this theorem is included in exercise \ref{ex:exercise_6.5}. A tighter bound is presented in exercise \ref{ex:exercise_6.6}.
\end{proof*}


















\newpage
\subsection{End of chapter exercises}
\begin{exercise}{}{exercise_6.1}
    Can you improve the constants in the empirical Rademacher complexity bound:
    \begin{align*}
        \sup_{g\in\G}\biggCurl{\E\bigSquare{g(Z)} - \frac{1}{n}\sum_{i=1}^n g(Z_i)} \le 2\ERC_S(\G) + 3(b-a)\sqrt{
            \frac{\log 2/\delta}{2n}
        }
    \end{align*}
    \noindent through a single, direct application of the bounded difference inequality?
\end{exercise}

\begin{solution*}[Exercise \ref{ex:exercise_6.1} - \color{red}{Wrong, to be fixed later} \color{black}]
    Given the sample $S=\{Z_1, \dots, Z_i, \dots, Z_n \}$ and define $S_i = \{Z_1, \dots, Z_i', \dots, Z_n \}$ for $1\le i \le n$. Define the following function:
    \begin{align*}
        \phi(S) &= \widehat{\E_S}[g] - \ERC_S(\G) \\
            &= \frac{1}{n}\sum_{i=1}^n g(Z_i) - \E_\sigma\biggSquare{
                \sup_{g\in\G}\frac{1}{n} \sum_{i=1}^n \sigma_i g(Z_i)
            }
    \end{align*}

    \noindent Note that we have:
    \begin{align*}
        \E[\phi(S)] &= \E\bigSquare{
            \widehat{\E_S}[g] - \ERC_S(\G)
        } = \E[g] - \RC_n(\G)
    \end{align*}

    \noindent We now check the bounded difference property of $\phi(S)$, for $1\le i \le n$, we have:
    \begin{align*}
        |\phi(S) - \phi(S_i)| &= \biggAbs{
            \frac{1}{n}\bigRound{g(Z_i) - g(Z_i')} - \E_\sigma\biggSquare{
                \sup_{g\in\G} \frac{1}{n}\sigma_i\bigRound{g(Z_i) - g(Z_i')}
            }
        } \\
        &= \biggAbs{
            \frac{1}{n}\bigRound{g(Z_i) - g(Z_i')} - \E_\sigma\biggSquare{
                \sup_{g\in\G} \frac{1}{n}\bigRound{g(Z_i) - g(Z_i')}
            }
        } \ \ \ (\text{Since $\sigma_i\in\{-1, 1\}$}) \\
        &\le 2\cdot\frac{b-a}{n}
    \end{align*}

    \noindent Hence, by the bounded difference inequality, let $\epsilon > 0$, we have:
    \begin{align*}
        P\bigRound{
            \phi(S) - \E[\phi(S)] \ge \epsilon
        } &\le \exp\biggRound{
            -\frac{2\epsilon^2}{\sum_{i=1}^n 4(b-a)^2/n^2}
        } \\
        &= \exp\biggRound{
            -\frac{n\epsilon^2}{2(b-a)^2}
        }
    \end{align*}

    \noindent Letting $\delta = \exp\biggRound{-\frac{n\epsilon^2}{2(b-a)^2}}$, we have:
    \begin{align*}
        \epsilon &= (b-a)\sqrt{
            \frac{2\log 1/\delta}{n}
        }
    \end{align*}

    \noindent Therefore, with probability of at least $1-\delta$ ($\delta>0$), we have:
    \begin{align*}
        \phi(S) - \E[\phi(S)] &\le (b-a)\sqrt{
            \frac{2\log 1/\delta}{n}
        } \\
        \implies
        \widehat{\E_S}[g] - \E[g] &\le (\ERC_S(\G) - \RC_n(\G)) + (b-a)\sqrt{
            \frac{2\log 1/\delta}{n}
        }
    \end{align*}
\end{solution*}

\begin{exercise}{}{exercise_6.2}
    \textbf{Definition} (Partition classifier) : Let $\Pi=\bigCurl{A_1, \dots, A_k}$ be a fixed partition of $\X$. The partition classifiers assign labels that corresponds to the partitions. Specifically, we define the following set of classifiers:
    \begin{align*}
        \Hf = \biggCurl{
            h : \X \to \{a_1, \dots, a_k\} \Bigg| h(x) = \sum_{j=1}^k a_j \1{x\in A_j}, \ a_j \in \{-1,1\}
        }
    \end{align*}

    \noindent Then we have $|\Hf|=2^k$. Derive the exact empirical Rademacher complexity of $\Hf$.
\end{exercise}

\begin{solution*}[Exercise \ref{ex:exercise_6.2}]
    We have:
    \begin{align*}
        \ERC_S(\Hf) 
            &= \E_\sigma\biggSquare{
                \sup_{h\in \Hf} \frac{1}{n} \sum_{i=1}^n \sigma_i h(X_i)
            } \\
            &= \frac{1}{n} \E_\sigma\biggSquare{
                \sup_{a_1, \dots, a_k} \sum_{i=1}^n \sigma_i \sum_{j=1}^k a_j \1{X_i\in A_j}
            } \\
            &= \frac{1}{n} \E_\sigma\biggSquare{
                \sup_{a_1, \dots, a_k} \sum_{j=1}^k \sum_{i=1}^n \sigma_i a_j \1{X_i\in A_j}
            } \\
            &= \frac{1}{n} \E_\sigma\biggSquare{
                \sup_{a_1, \dots, a_k} \sum_{j=1}^k \sum_{i\in[n]; X_i \in A_j} \sigma_i a_j
            } \\
            &= \frac{1}{n} \E_\sigma\biggSquare{
                \sum_{j=1}^k \sup_{a_j\in\{-1, 1\}} \sum_{i\in[n]; X_i \in A_j} \sigma_i a_j
            } \\
            &= \frac{1}{n}\sum_{j=1}^k \E_\sigma\biggSquare{
                \sup_{a_j\in\{-1, 1\}} a_j \sum_{i\in[n]; X_i \in A_j} \sigma_i
            }
    \end{align*}

    \noindent Given a sample of Rademacher variables, to maximize the sum of those variables, the strategy is to multiply the entire sample with the dominant sign. For example,
    \begin{align*}
        \{-1, -1, 1\} &\to \{1, 1, -1\} &&\Sigma = 1 \\
        \{1, 1, -1\} &\to \{1, 1, -1\} &&\Sigma = 1
    \end{align*}

    \noindent This is equivalent to taking the absolute value of the sum. Hence, we have:
    \begin{align*}
        \ERC_S(\Hf) 
            &= \frac{1}{n}\sum_{j=1}^k \E_\sigma\biggSquare{
                \biggAbs{
                    \sum_{i\in[n]; X_i \in A_j} \sigma_i
                }
            } \\
            &= \frac{1}{n}\sum_{j=1}^k \E_\sigma\biggSquare{
                \biggAbs{
                    \sum_{i \in \mathcal{N}_j} \sigma_i
                }
            }, \ \ \ \mathcal{N}_j = \bigCurl{ i : X_i \in A_j } 
    \end{align*}

    \noindent We can model $\bigAbs{\sum_{i\in \mathcal{N}_j}\sigma_i}$ as the absolute difference between two binomial variables with distribution $Binomial(n_j, p=\frac{1}{2})$ where $n_j=|\mathcal{N}_j|$. For each $j\in \{1, \dots, k\}$, denote that:
    \begin{align*}
        \biggAbs{
            \sum_{i \in \mathcal{N}_j} \sigma_i
        } &= \bigAbs{ X_j - Y_j } \text{ where } X_j, Y_j \sim Binomial\Bigg(n_j, p=\frac{1}{2}\Bigg)
    \end{align*}

    \begin{subproof}{Compute $P\bigRound{\bigAbs{X_j - Y_j} = z}$ for $z\ne 0$}
        For the case when $X_j \ne Y_j$, we have:
        \begin{align*}
            P\bigRound{\bigAbs{X_j - Y_j} = z} 
                &= P\bigRound{X_j = z + Y_j} + P\bigRound{Y_j = z + X_j}
        \end{align*}

        \noindent Since the two events are independent and has the same probability (due to the fact that $X_j$ and $Y_j$ are identically distributed). By the law of total probability, we have:
        \begin{align*}
            P\bigRound{X_j = z + Y_j}
                &= \sum_{y=0}^{n_j - z} P(X_j = z + Y_j | Y_j = y)P(Y_j = y) \\
                &= \sum_{y=0}^{n_j - z} \biggSquare{\begin{pmatrix} n_j \\ z + y \end{pmatrix} \frac{1}{2^{n_j}}} \biggSquare{ \begin{pmatrix} n_j \\ y \end{pmatrix} \frac{1}{2^{n_j}} } \\
                &= \frac{1}{2^{2n_j}}\sum_{y=0}^{n_j-z} \begin{pmatrix} n_j \\ y \end{pmatrix} \begin{pmatrix} n_j \\ z + y \end{pmatrix} \\
                &= \frac{1}{2^{2n_j}}\sum_{y=0}^{n_j-z} \begin{pmatrix} n_j \\ y \end{pmatrix} \begin{pmatrix} n_j \\ n_j - z - y \end{pmatrix} \\
                &= \frac{1}{2^{2n_j}}\begin{pmatrix}
                    2n_j \\ n_j - z
                \end{pmatrix} \ \ \ \text{(Vandermonde's Identity)}
        \end{align*}

        \noindent Hence, for $X_j\ne Y_j$, we have:
        \begin{align*}
            P\bigRound{\bigAbs{X_j - Y_j} = z} &= 2 \times \frac{1}{2^{2n_j}}\begin{pmatrix}
                2n_j \\ n_j - z
            \end{pmatrix} \\
            &= \frac{1}{2^{2n_j-1}}\begin{pmatrix}
                2n_j \\ n_j - z
            \end{pmatrix}
        \end{align*}
    \end{subproof}

    \begin{subproof}{Compute $\E\bigSquare{\bigAbs{X_j - Y_j}}$}
        Now that we have the PMF for $\bigAbs{X_j - Y_j}$, we have:
        \begin{align*}
            \E\bigSquare{\bigAbs{X_j - Y_j}}
                &= \sum_{z=1}^{n_j} z \times P\bigRound{\bigAbs{X_j - Y_j} = z} \\
                &= \frac{1}{2^{2n_j - 1}} \sum_{z=1}^{n_j} z \times \begin{pmatrix}
                    2n_j \\ n_j - z
                \end{pmatrix}
        \end{align*}
    \end{subproof}

    \noindent Now, plug the above into the formula of the empirical Rademacher Complexity, we have:
    \begin{align*}
        \ERC_S(\Hf)
            &= \frac{1}{n}\sum_{j=1}^k \E_\sigma\biggSquare{\biggAbs{
                \sum_{i\in\mathcal{N}_j}\sigma_i
            }} \\
            &= \frac{1}{n}\sum_{j=1}^k \E\bigSquare{\bigAbs{
                X_j - Y_j
            }} \\
            &= \frac{1}{n}\sum_{j=1}^k \frac{1}{2^{2n_j - 1}} \sum_{z=1}^{n_j} z \times \begin{pmatrix}
                2n_j \\ n_j - z
            \end{pmatrix}
    \end{align*}
\end{solution*}

