\newpage
\section{Bayes classifier}

\subsection{Properties of Bayes Risk}
\textbf{Overview} : Recall that the Bayes classifier is the one with minimum risk and the corresponding risk is called the Bayes Risk. For $\mathcal{Y} = \{0,1 \}$ and defined:

\begin{align*}
    \eta(x) = P(Y=1|X=x)
\end{align*}

\noindent Define the following classifier:
\begin{align*}
    h^*(x) = \begin{cases}
        1 & \text{if } \eta(x) \ge \frac{1}{2}
        \\ \\
        0 & \text{otherwise}
    \end{cases}
\end{align*}

\begin{theorem}{Properties of Bayes classifier}{properties_of_bayes_classifier}
    The following properties hold for the Bayes classifier with $\mathcal{Y} = \{0,1\}$ (Binary classification):
    \begin{itemize}
        \item $(i)$ $R(h^*) = \inf_{h:\mathcal{X}\to\mathcal{Y}}\{ R(h) \} = R^*$.
        \item $(ii)$ $\underbrace{R(h) - R^*}_{\text{Excess risk}} = 2\mathbb{E}_X\Bigg[ \Big| \eta(x) - \frac{1}{2}\Bigg| \1{h(X)\ne h^*(X)} \Bigg]$.
        \item $(iii)$ $R^* = \mathbb{E}\Big[ \min(\eta(X), 1 - \eta(x)) \Big]$.
    \end{itemize}
\end{theorem}

\begin{proof*}[Theorem \ref{thm:properties_of_bayes_classifier}]
    Proving each point:

    \begin{subproof}{\newline $(i)$ $R(h^*) = \inf_{h:\mathcal{X}\to\mathcal{Y}}\{ R(h) \} = R^*$.}
        For all $h:\mathcal{X} \to \mathcal{Y}$, we have:
        \begin{align*}
            R(h) 
                &= \mathbb{E}_{XY} \Big[ \1{h(X) \ne Y} \Big] \\
                &= \mathbb{E}_{x\sim X}\Bigg[
                    \mathbb{E}_{Y|X=x} \Big[ \1{Y \ne h(x)} \Big]
                \Bigg] \\
                &= \mathbb{E}_{x\sim X}\Bigg[
                    \sum_{y\in\{0,1\}}\1{y \ne h(x)}
                \Bigg] \\
                &= \mathbb{E}_{x\sim X}\Big[ \eta(x)\1{h(x)=0} + (1 - \eta(x)) \1{h(x)=1} \Big]
        \end{align*}

        \noindent Since the two events $\{h(x)=1\}$ and $\{h(x)=0\}$ are mutually exclusive, $R(h)$ is the smallest when we set $h(x)=1$ when $\eta(x) \ge 1 - \eta(x)\implies \eta(x) \ge \frac{1}{2}$. Therefore, we have:
        \begin{align*}
            h^*(x) = \begin{cases}
                1 & \text{if } \eta(x) \ge \frac{1}{2}
                \\ \\
                0 & \text{otherwise}
            \end{cases}
        \end{align*} 
    \end{subproof}

    \begin{subproof}{\newline $(ii)$ $\underbrace{R(h) - R^*}_{\text{Excess risk}} = 2\mathbb{E}_X\Bigg[ \Big| \eta(x) - \frac{1}{2}\Bigg| \1{h(X)\ne h^*(X)} \Bigg]$.}
        We have:
        \begin{align*}
            R(h) - R^* 
                &= \mathbb{E}_{x\sim X}\Bigg[
                    \mathbb{E}_{Y|X=x} \Big[ \1{Y \ne h(x)} \Big]
                \Bigg] - \mathbb{E}_{x\sim X}\Bigg[
                    \mathbb{E}_{Y|X=x} \Big[ \1{Y \ne h^*(x)} \Big]
                \Bigg] \\
                &= \mathbb{E}_{x\sim X}\Bigg[
                    \sum_{y\in\{0,1\}}\1{y \ne h(x)}P(Y=y|X=x)
                \Bigg] - \mathbb{E}_{x\sim X}\Bigg[
                    \sum_{y\in\{0,1\}}\1{y \ne h^*(x)}P(Y=y|X=x)
                \Bigg] \\
                &= \mathbb{E}_{x\sim X}\Bigg[
                    \eta(x)\Big( \1{h(x) = 0} - \1{h^*(x) = 0} \Big)
                    + (1 - \eta(x)) \Big( \1{h(x)=1} - \1{h^*(x) = 1} \Big)
                \Bigg] \\
                &= \mathbb{E}_{x\sim X}\Bigg[
                    \eta(x)\Big( \1{h(x) \ne h^*(x), h(x)=0} - \1{h(x) \ne h^*(x), h(x)=1} \Big) \\
                    & \ \ \ \ \ \ \ \ \ 
                    + (1 - \eta(x)) \Big( \1{h(x) \ne h^*(x), h(x)=1} - \1{h(x) \ne h^*(x), h(x)=0} \Big) 
                \Bigg] \\
                &= \mathbb{E}_{x\sim X}\Bigg[
                    (2\eta(x) - 1)\1{h(x) \ne h^*(x), h(x)=0} + (1 - 2\eta(x))\1{h(x) \ne h^*(x), h(x)=1}
                \Bigg] \\
                &= \mathbb{E}_{x\sim X}\Bigg[
                    \Big| 2\eta(x) - 1 \Big| \1{h(x) \ne h^*(x)}
                \Bigg] \\
                &= 2\mathbb{E}_X\Bigg[\Bigg| \eta(X) - \frac{1}{2} \Bigg|\1{h(X)\ne h^*(X)}\Bigg]
        \end{align*}
    \end{subproof}

    \begin{subproof}{\newline $(iii)$ $R^* = \mathbb{E}\Big[ \min(\eta(X), 1 - \eta(x)) \Big]$.}
        From $(i)$ we have:
        \begin{align*}
            R(h^*) &= \mathbb{E}_{x\sim X}\Big[ \eta(x)\1{h^*(x)=0} + (1 - \eta(x)) \1{h^*(x)=1} \Big] \\
            &= \mathbb{E}_X\Big[ \min(\eta(X), 1 - \eta(x)) \Big]
        \end{align*}
    \end{subproof}
\end{proof*}

\begin{theorem}{Properties of Bayes classifier (Multi-class)}{properties_of_bayes_classifier_multiclass}
    For multi-class classification with more than two labels : $\mathcal{Y}=\{1, 2, \dots, M\}$, the Bayes classifier is defined as followed:
    \begin{align*}
        h^*(x) &= \arg\max_{y\in\mathcal{Y}}\Big\{ \eta_y(x) \Big\} \\
        \text{Where : } \eta_y(x) &= P(Y=y|X=x)
    \end{align*}
    
    \noindent\newline The following properties hold for the Bayes classifier with $\mathcal{Y}=\{1, 2, \dots, M\}$ (Multi-class classification):
    \begin{itemize}
        \item $(i)$ \textbf{Bayes Risk $R^*$} :
        \begin{align*}
            R^* = \mathbb{E}_{x\sim X}\Big[ 1 - \max_{y\in\mathcal{Y}}\Big\{ \eta_y(x) \Big\}\Big] = \mathbb{E}_{x\sim X}\Big[\min_{y\in\mathcal{Y}} \overline{\eta_y}(x) \Big]
        \end{align*} 

        \item $(ii)$ \textbf{Excess Risk $R(h)-R^*$} :
        \begin{align*}
            R(h) - R^* = \mathbb{E}_X\Big[ \Big( \eta_{y^*_x}(x) - \eta_{y_x}(x) \Big) \1{h(x)\ne h^*(x)} \Big]
        \end{align*}

        \noindent Where $y_x=h(x)$ is the prediction made by an arbitrary classifier $h:\mathcal{X}\to\mathcal{Y}$ and $y^*_x=h^*(x)$ is the prediction made by the Bayes classifier.
    \end{itemize}
\end{theorem}

\begin{proof*}[Theorem \ref{thm:properties_of_bayes_classifier_multiclass}]
    (The proof of this theorem has been included in the solution of Exercise \ref{ex:exercise_2.1}).
\end{proof*}

\subsection{Likelihood Ratio Test}
\textbf{Overview} : Define $\pi_1 = P(Y=1)$ and $\pi_0 = P(Y=0)$ be the prior probabilities. Let $p_1(x)=P(X=x|Y=1)$ and $p_0(x)=P(X=x|Y=0)$ be the class-conditional densities. Note that we have:
\begin{align*}
    \eta(x) &= P(Y=1|X=x) \\
        &= \frac{P(X=x|Y=1)P(Y=1)}{P(X=x|Y=1)P(Y=1) + P(X=x|Y=0)P(Y=0)} \\
        &= \frac{\pi_1p_1(x)}{\pi_1p_1(x) + \pi_0p_0(x)} \\
        &= \frac{1}{1 + \frac{\pi_0p_0(x)}{\pi_1p_1(x)}}
\end{align*}

\noindent\newline Hence, we have:
\begin{align*}
    \eta(x) \ge \frac{1}{2} &\iff \frac{\pi_0p_0(x)}{\pi_1p_1(x)} \\
        &\iff \frac{p_1(x)}{p_0(x)} \ge \frac{\pi_0}{\pi_1}
\end{align*}

\begin{proposition}{Likelihood ratio test}{likelihood_ratio_test}
    The Bayes classifier $h^*$ can be re-defined as followed:
    \begin{align*}
        h^*(x) = \begin{cases}
            1 & \text{if } \frac{p_1(x)}{p_0(x)} \ge \frac{\pi_0}{\pi_1}
            \\ \\
            0 & \text{otherwise}
        \end{cases}
    \end{align*}

    \noindent The fraction $\frac{p_1(x)}{p_0(x)}$ is called the \textbf{likelihood ratio}.
\end{proposition}


\subsection{Plug-in classifier}
\begin{definition}[Plug-in classifier]
    A \textbf{plug-in classifier} is based on an estimate of $\eta(x)$. This estimate is then plugged into the definition of the Bayes classifier. Suppose that $\widehat{\eta_n}$ is an estimate of $\eta$ based on $n$ training samples $\{(X_i, Y_i)\}_{i=1}^n$. We define $\widehat{h_n}$ as:
    \begin{align*}
        \widehat{h_n} = \begin{cases}
            1 & \text{if } \widehat{\eta_n}(x) \ge \frac{1}{2}
            \\ \\
            0 & \text{otherwise}
        \end{cases}
    \end{align*}
\end{definition}

\begin{corollary}{Excess risk of plug-in classifier}{excess_risk_of_plugin_classifier}
    We have the following upper-bound for the excess risk of the plug-in classifier:
    \begin{align*}
        R(\widehat{h_n}) - R^* \le 2 \mathbb{E}_X\Big[ \Big| \eta(X) - \widehat{\eta_n}(X) \Big| \Big]
    \end{align*}
\end{corollary}

\begin{proof*}[Corollary \ref{coro:excess_risk_of_plugin_classifier}]
    From theorem \ref{thm:properties_of_bayes_classifier}, we have:
    \begin{align*}
        R(\widehat{h_n}) - R^* &= 2\mathbb{E}_X\Bigg[ \Big| \eta(X) - \frac{1}{2}\Big|\1{\widehat{h_n}(X)\ne h^*(X)}\Bigg]
    \end{align*}

    \noindent The indicator term will be non-zero in the above equality if one of the following cases occurs:
    \begin{align*}
        \begin{cases}
            \widehat{h_n}(X) = 1, h^*(X) = 0
            \\ \\
            \widehat{h_n}(X) = 0, h^*(X) = 1
        \end{cases}
        \implies 
        \begin{cases}
            \widehat{\eta_n}(X) \ge \frac{1}{2}, \eta(X) < \frac{1}{2}
            \\ \\
            \widehat{\eta_n}(X) < \frac{1}{2}, \eta(X) \ge \frac{1}{2}
        \end{cases}
    \end{align*}

    \begin{subproof}{\newline Case 1 : $\widehat{\eta_n}(X) \ge \frac{1}{2}, \eta(X) < \frac{1}{2}$}
        We have:
        \begin{align*}
            \eta(X) - \widehat{\eta_n}(X) &\le \eta(X) - \frac{1}{2} \ \ \ (\text{Both sides negative}) \\
            \implies \Bigg| \eta(X) - \widehat{\eta_n}(X) \Bigg| &\ge \Bigg| \eta(X) - \frac{1}{2} \Bigg|
        \end{align*}
    \end{subproof}

    \begin{subproof}{\newline Case 2 : $\widehat{\eta_n}(X) < \frac{1}{2}, \eta(X) \ge \frac{1}{2}$}
        We have:
        \begin{align*}
            \widehat{\eta_n}(X) - \eta(X) \ge \widehat{\eta_n}(X) - \frac{1}{2} \ge  \eta(X) - \frac{1}{2} \ \ \ \text{(All positive)}
        \end{align*}

        \noindent\newline Therefore, we have:
        \begin{align*}
            \Bigg| \eta(X) - \widehat{\eta_n}(X) \Bigg| &\ge \Bigg| \eta(X) - \frac{1}{2} \Bigg|
        \end{align*}
    \end{subproof}
    \noindent\newline For both cases, we have the same $\Big| \eta(X) - \widehat{\eta_n}(X) \Big| \ge \Big| \eta(X) - \frac{1}{2} \Big|$ inequality. Therefore, we have:
    \begin{align*}
         R(\widehat{h_n}) - R^* \le 2 \mathbb{E}_X\Big[ \Big| \eta(X) - \widehat{\eta_n}(X) \Big| \Big]
    \end{align*}
\end{proof*}

\newpage
\subsection{End of chapter exercises}
\begin{exercise}{}{exercise_2.1}
    Extend theorem \ref{thm:properties_of_bayes_classifier} to the multi-class classification case where $\mathcal{Y}=\{1, 2, \dots, M\}$. In other words, prove theorem \ref{thm:properties_of_bayes_classifier_multiclass}.
\end{exercise}

\begin{solution*}[Exercise \ref{ex:exercise_2.1}]
    We re-define the Bayes classifier $h^*$ as followed:
    \begin{align*}
        h^*(x) &= \arg\max_{y\in\mathcal{Y}} \Big\{ \eta_y(x) \Big\}, \\ 
        \eta_y(x) &= P(Y=y|X=x)
    \end{align*}

    \noindent We have:
    \begin{align*}
        \sum_{y\in\mathcal{Y}} \eta_y(x) = 1, \ \forall x \in \mathcal{X}
    \end{align*}

    \begin{subproof}{$\bf (i)$ Calculate Bayes risk $R^*$}
        For any classifier $h:\mathcal{X} \to \mathcal{Y}$, we have:
        \begin{align*}
            R(h) &= \mathbb{E}_{x\sim X}\Bigg[
                \sum_{y\in\mathcal{Y}} \1{h(x) \ne y}\eta_y(x)
            \Bigg]
        \end{align*}

        \noindent Letting $\hat y_x = h(x)$ being $h$'s prediction for a given feature vector $x\in\mathcal{X}$, we have:
        \begin{align*}
            R(h) &= \mathbb{E}_{x\sim X}\Bigg[
                \sum_{y\in\mathcal{Y};y\ne \hat y_x} \eta_y(x)
            \Bigg]
            = \mathbb{E}_{x\sim X}\Bigg[
                1 - \eta_{\hat y_x}(x)
            \Bigg]
        \end{align*}

        \noindent In order to minimize $R(h)$, we need $\eta_{\hat y_x}(x)$ to be maxmized for all $x\in\mathcal{X}$. Hence, we have:
        \begin{align*}
            R^* = \mathbb{E}_{x\sim X}\Bigg[
                1 - \max_{y\in\mathcal{Y}} \Big\{ \eta_y(x) \Big\}
            \Bigg]
        \end{align*}

        \noindent Therefore, we have $h^*(x) = \arg\max_{y\in\mathcal{Y}} \Big\{ \eta_y(x) \Big\}$ is the Bayes classifier and the Bayes risk $ R^* = \mathbb{E}_{x\sim X}\Bigg[1 - \max_{y\in\mathcal{Y}} \Big\{ \eta_y(x) \Big\}\Bigg]$.
    \end{subproof}

    \begin{subproof}{\newline $\bf (ii)$ Calculate excess risk $R(h) - R^*$}
        For any $h:\mathcal{X}\to\mathcal{Y}$, we have:
        \begin{align*}
            R(h) - R^* 
                &= \mathbb{E}_{x\sim X}\Bigg[
                    \sum_{y\in\mathcal{Y}} \1{h(x) \ne y}\eta_y(x)
                \Bigg] - \mathbb{E}_{x\sim X}\Bigg[ 1 - \max_{y\in\mathcal{Y}}\Big\{ \eta_y(x) \Big\}\Bigg] \\
                &= \mathbb{E}_{x\sim X}\Bigg[
                    \sum_{y\in\mathcal{Y}} \1{h(x) \ne y}\eta_y(x) + \max_{y\in\mathcal{Y}}\Big\{ \eta_y(x) \Big\} - 1
                \Bigg]
        \end{align*}

        \noindent Denote $h^*(x) = y^*_x$ and $h(x)=y_x$. When $h(x) = h^*(x) = y^*_x$, we have:
        \begin{align*}
            \sum_{y\in\mathcal{Y}} \1{h(x) \ne y}\eta_y(x) + \max_{y\in\mathcal{Y}}\Big\{ \eta_y(x) \Big\} 
                &=\sum_{y\in\mathcal{Y}; y\ne y_x}\eta_y(x) + \eta_{y^*_x}(x) \\
                &=\sum_{y\in\mathcal{Y}; y\ne y^*_x}\eta_y(x) + \eta_{y^*_x}(x) \\
                &=\sum_{y\in\mathcal{Y}}\eta_y(x)=1 \\
            \implies \sum_{y\in\mathcal{Y}} \1{h(x) \ne y}\eta_y(x) + \max_{y\in\mathcal{Y}}\Big\{ \eta_y(x) \Big\} - 1 &= 0
        \end{align*}

        \noindent\newline When $h(x) \ne h^*(x)$, we have:
        \begin{align*}
            \sum_{y\in\mathcal{Y}} \1{h(x) \ne y}\eta_y(x) + \max_{y\in\mathcal{Y}}\Big\{ \eta_y(x) \Big\} - 1
                &= \sum_{y\in\mathcal{Y}; y\ne y_x}\eta_y(x) + \eta_{y^*_x}(x) - 1 \\
                &= 2\eta_{y^*_x}(x) - 1 + \sum_{y\in\mathcal{Y}\setminus\{y_x, y_x^*\}}\eta_y(x) \\
                &= 2\eta_{y^*_x}(x) - \Big( \eta_{y_x}(x) + \eta_{y_x^*}(x) \Big) \\
                &= \eta_{y^*_x}(x) - \eta_{y_x}(x)
                .
        \end{align*}

        \noindent Therefore, we can re-write the excess risk by multiplying the entire integrand with the indicator function $\1{h(x)\ne h^*(x)}$ as followed:

        \begin{align*}
             R(h) - R^*  = \mathbb{E}_{x\sim X}\Bigg[ \Big( \eta_{y^*_x}(x) - \eta_{y_x}(x) \Big) \1{h(x)\ne h^*(x)} \Bigg]
        \end{align*}
    \end{subproof}

    \begin{subproof}{$\bf (iii)$ Simpler form of Bayes risk}
        From $(i)$ we have:
        \begin{align*}
            R^* = \mathbb{E}_X\Big[ 1 - \max_{y\in\mathcal{Y}}\Big\{ \eta_y(x) \Big\} \Big] = \mathbb{E}_X\Big[\min_{y\in\mathcal{Y}} \Big\{ \overline{\eta_y}(x) \Big\} \Big]
        \end{align*}

        \noindent Where $\overline{\eta_y}(x)=P(Y\ne y|X=x)$.
    \end{subproof}
\end{solution*}

\begin{exercise}{}{exercise_2.2}
    Define the \textbf{$\alpha$-cost-sensitive risk} of a classifier $h:\mathcal{X}\to\mathcal{Y}$ as followed:
    \begin{align*}
        R_\alpha(h) = \mathbb{E}_{XY}\Big[ (1-\alpha)\1{Y=1, h(X)=0} + \alpha\1{Y=0, h(X)=1} \Big]
    \end{align*}

    \noindent Define the Bayes classifier and prove and analogue of theorem \ref{thm:properties_of_bayes_classifier}.
\end{exercise}

\begin{solution*}[Exercise \ref{ex:exercise_2.2}]
    Using the law of total expectation, we have:
    \begin{align*}
        R_\alpha(h)
            &= \mathbb{E}_{x\sim X}\Bigg[
                \sum_{y\in\{0,1\}}\Big[(1-\alpha)\1{y=1, h(x)=0} + \alpha\1{y=0, h(x)=1} \Big] P(Y=y|X=x) 
            \Bigg] \\
            &= \mathbb{E}_{x\sim X}\Big[
                (1-\alpha)\eta(x)\1{h(x)=0} + \alpha(1-\eta(x))\1{h(x)=1}
            \Big]
    \end{align*}

    \noindent Since $\1{h(x)=0}$ and $\1{h(x)=1}$ are mutually exclusive, in order for $R_\alpha(h)$ to be minimize, we define the following Bayes classifier:
    \begin{align*}
        h^*(x) = \begin{cases}
            1 &\text{if } \alpha(1-\eta(x)) \le (1-\alpha)\eta(x)
            \\ \\
            0 &\text{otherwise}
        \end{cases}
        = \begin{cases}
            1 &\text{if } \eta(x) \ge \alpha
            \\ \\
            0 &\text{otherwise}
        \end{cases}
    \end{align*}

    \noindent We can also derive a likelihood-ratio test version of the Bayes classifier, we have:
    \begin{align*}
        \eta(x) \ge \alpha &\implies \frac{1}{1+\frac{\pi_0p_0(x)}{\pi_1p_1(x)}} \ge \alpha \\
            &\implies 1 + \frac{\pi_0\cdot p_0(x)}{\pi_1\cdot p_1(x)}\le \frac{1}{\alpha} \\
            &\implies \frac{p_1(x)}{p_0(x)} \ge \frac{\alpha}{1-\alpha}\cdot \frac{\pi_0}{\pi_1}
    \end{align*}

    \noindent Hence, we can rewrite the Bayes classifier as followed:
    \begin{align*}
        h^*(x) = \begin{cases}
            1 &\text{if } \frac{p_1(x)}{p_0(x)} \ge \frac{\alpha}{1-\alpha}\cdot \frac{\pi_0}{\pi_1} 
            \\ \\
            0 &\text{otherwise}
        \end{cases}
    \end{align*}

    \begin{subproof}{\newline $\bf(i)$ Bayes Risk $R_\alpha^*$}
        We have:
        \begin{align*}
            R_\alpha^* &= R_\alpha(h^*) \\
                &= \mathbb{E}_{x\sim X}\Big[
                    (1-\alpha)\eta(x)\1{h^*(x)=0} + \alpha(1-\eta(x))\1{h^*(x)=1}
                \Big] \\
                &= \mathbb{E}_X\Big[ \min(\alpha(1-\eta(X)), (1-\alpha)\eta(X)) \Big]
        \end{align*}
    \end{subproof}

    \begin{subproof}{\newline $\bf (ii)$ Excess Risk $R_\alpha(h)-R_\alpha^*$}
        For an arbitrary $h:\mathcal{X} \to\mathcal{Y}$, we have:
        \begin{align*}
            R_\alpha(h) - R_\alpha^* 
                &=  \mathbb{E}_{x\sim X}\bigSquare{
                    (1-\alpha)\eta(x)\bigRound{\1{h(x)=0} - \1{h^*(x)=0}} +\alpha(1-\eta(x))\bigRound{\1{h(x)=1} - \1{h^*(x)=1}}
                } \\
                &= \mathbb{E}_{x\sim X}\bigSquare{
                    (1-\alpha)\eta(x)\bigRound{\1{h(x)=0, h^*(x)=1} - \1{h(x)=1, h^*(x)=0}} \\
                & \ \ \ \ +\alpha(1-\eta(x))\bigRound{\1{h(x)=1, h^*(x)=0} - \1{h(x)=0, h^*(x)=1}}
                } \\
                &= \mathbb{E}_{x\sim X}\bigSquare{
                    \1{h(x)=0, h^*(x)=1}(\eta(x)-\alpha) + \1{h(x)=1, h^*(x)=0}(\alpha-\eta(x))
                } \\
                &= \mathbb{E}_{X}\bigSquare{
                    \bigAbs{\eta(X) - \alpha}\1{h(X)\ne h^*(X)}
                }
        \end{align*}
    \end{subproof}
\end{solution*}



