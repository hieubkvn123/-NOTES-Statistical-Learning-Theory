\newpage\section{Hoeffding's inequality}

\subsection{Markov's Inequality}
\begin{proposition}{Markov's Inequality}{markov_ineq}
    Let $U$ be a non-negative random variable on $\mathbb{R}$, then for all $t>0$, we have:
    \begin{align*}
        P(U \ge t) \le \frac{1}{t}\mathbb{E}[U]
    \end{align*}
\end{proposition}

\begin{proof*}[Proposition \ref{prop:markov_ineq}]
    We have:
    \begin{align*}
        tP(U \ge t) 
            &= t\mathbb{E}\bigSquare{\1{U\ge t}} \\
            &= t\int_0^\infty \1{x \ge t} f_U(x)dx \\
            &= t\int_t^\infty f_U(x)dx \\
            &\le \int_t^\infty xf_U(x)dx \\
            &\le \int_0^\infty xf_U(x)dx = \mathbb{E}[U] \\
        \implies P(U\ge t) &\le \frac{1}{t}\mathbb{E}[U]
    \end{align*}
\end{proof*}

\begin{corollary}{Chebyshev's Inequality}{chebyshev_ineq}
    Let $Z$ be a random variable on $\mathbb{R}$ with mean $\mu$ and variance $\sigma^2$, we have:
    \begin{align*}
        P\bigRound{
            \bigAbs{
                Z - \mu
            } \ge t
        } \le \frac{\sigma^2}{t^2}
    \end{align*}
\end{corollary}

\begin{proof*}[Corollary \ref{coro:chebyshev_ineq}]
    Using Markov's inequality, we have:
    \begin{align*}
        P\bigRound{
            \bigAbs{
                Z - \mu
            } \ge t
        } &= 
        P\bigRound{
            \bigAbs{
                Z - \mu
            }^2 \ge t^2
        } \\
        &\le \frac{\mathbb{E}\bigSquare{\bigAbs{Z-\mu}^2}}{t^2} = \frac{\sigma^2}{t^2}
    \end{align*}
\end{proof*}

\begin{corollary}{Chernoff's bounding method}{chernoff_bound}
    Let $Z$ be a random variable on $\mathbb{E}$, for any $t>0$, we have:
    \begin{align*}
        P(Z\ge t) \le \inf_{s>0} e^{-st}M_Z(s)
    \end{align*}
\end{corollary}

\begin{proof*}[Corollary \ref{coro:chernoff_bound}]
    We have:
    \begin{align*}
        P(Z \ge t) &= P(sZ \ge st), \ \ \ (t > 0) \\
            &= P(e^{sZ} \ge e^{st}) \\
            &\le \frac{\mathbb{E}\bigSquare{e^{sZ}}}{e^{st}} = e^{-st}M_Z(s) \ \ \ (\text{Markov's inequality})
    \end{align*}

    \noindent Since the above inequality holds for all $s>0$, we can just take the infimum to obtain the tightest bound. Hence, we have:
    \begin{align*}
        P(Z \ge t) \le \inf_{s>0}e^{-st}M_Z(s)
    \end{align*}
\end{proof*}

\subsection{Hoeffding's Inequality}
Before diving into Hoeffding's inequality, we need to go through the following lemma (whose proof will not be included) that will help us prove the Hoeffding's inequality:
\begin{lemma}{Hoeffding's lemma}{hoeffding_lemma}
    Let $V$ be a random variable on $\mathbb{R}$ with $\mathbb{E}[V]=0$ and suppose that $a\le V \le b$ with probability one. We have:
    \begin{align*}
        \mathbb{E}\bigSquare{e^{sV}} \le \exp\biggRound{
        \frac{s^2(b-a)^2}{8}
        }
    \end{align*}
\end{lemma}
\begin{proof*}[Lemma \ref{lem:hoeffding_lemma}]
    (The proof for this lemma can be found here \cite{wiki:hoeffding_lemma}).
\end{proof*}

\begin{theorem}{Hoeffding's Inequality}{hoeffding_inequality}
    Let $Z_1, Z_2, \dots, Z_n$ be independent random variables on $\mathbb{R}$ such that $a_i \le Z_i \le b_i$ with probability one for all $1\le i \le n$. Let $S_n = \sum_{i=1}^n Z_i$. We have:
    \begin{align*}
        P\bigRound{ \bigAbs{ S_n - \mathbb{E}[S_n] } \ge t } \le 2\exp\biggRound{
        -\frac{2t^2}{\sum_{i=1}^n (b_i-a_i)^2}
        }, \ \ \ \forall t > 0
    \end{align*}
\end{theorem}

\begin{proof*}[Theorem \ref{thm:hoeffding_inequality}]
    Using the Chernoff's bounds, we have:
    \begin{align*}
        P\bigRound{S_n - \mathbb{E}[S_n] \ge t}
            &\le \inf_{s>0} e^{-st}M_{S_n - \mathbb{E}[S_n]}(s) \\
            &= \inf_{s>0}e^{-st}\mathbb{E}\bigSquare{
                e^{s(S_n - \mathbb{E}[S_n])}
            } \\
            &= \inf_{s>0}e^{-st}\mathbb{E}\biggSquare{
                \exp\biggRound{
                    s\sum_{i=1}^n (Z_i - \mathbb{E}[Z_i])
                }
            } \\
            &= \inf_{s>0}e^{-st}\mathbb{E}\biggSquare{
                \prod_{i=1}^n \exp\bigRound{s( Z_i - \mathbb{E}[Z_i]) }
            } \\
            &= \inf_{s>0}e^{-st} \prod_{i=1}^n \mathbb{E}\biggSquare{
                \exp\bigRound{s( Z_i - \mathbb{E}[Z_i]) }
            } \ \ \ (\text{Since all $Z_i - \mathbb{E}[Z_i]$ are independent}) \\
            &\le \inf_{s>0}e^{-st} \prod_{i=1}^n \exp\biggRound{
                \frac{s^2(b_i - a_i)^2}{8}
            } \ \ \ (\text{By Hoeffding's lemma}) \\
            &= \inf_{s>0} \exp\biggRound{
                -st + \sum_{i=1}^n \frac{s^2(b_i - a_i)^2}{8}
            }
    \end{align*}

    \noindent In order for the above to be minimized, we differentiate the term inside the exponential and set the derivative to $0$ to find the optimal $s>0$. We have:
    \begin{align*}
        -t + s\sum_{i=1}^n \frac{(b_i-a_i)^2}{4} = 0 \implies s = \frac{4t}{\sum_{i=1}^n(b_i - a_i)^2}
    \end{align*}

    \noindent Letting $c = \sum_{i=1}^n(b_i-a_i)^2$, we now can derive the tightest Chernoff's bound as followed:
    \begin{align*}
        P\bigRound{S_n - \mathbb{E}[S_n] \ge t}
            &\le \exp\biggRound{
                -\frac{4t^2}{c} + \frac{16t^2}{c^2} \cdot \frac{c}{8}
            } = \exp\biggRound{
                -\frac{2t^2}{c}
            } \\
            &= \exp\biggRound{
                -\frac{2t^2}{\sum_{i=1}^n(b_i - a_i)^2}
            }
    \end{align*}

    \noindent Repeating the same argument, we can also prove that:
    \begin{align*}
        P\bigRound{\mathbb{E}[S_n] - S_n \ge t} &\le \exp\biggRound{
            -\frac{2t^2}{\sum_{i=1}^n(b_i - a_i)^2}
        }
    \end{align*}

    \noindent Combining the two sides of the inequality, we have:
    \begin{align*}
        P\bigRound{\bigAbs{S_n - \mathbb{E}[S_n]} \ge t} &\le 2\exp\biggRound{
            -\frac{2t^2}{\sum_{i=1}^n(b_i - a_i)^2}
        }
    \end{align*}
\end{proof*}


\subsection{Convergence of Empirical Risk}
\begin{definition}[Empirical Risk ($\widehat{R_n}$)]
    Suppose we are given training data $\bigCurl{(X_i, Y_i)_{i=1}^n}$ such that each pair $(X_i, Y_i)\sim P_{XY}$ are independently identically distributed. Let $h:\mathcal{X}\to\mathcal{Y}$ be a classifier. We define the \textbf{empirical risk} to be:
    \begin{align*}
        \widehat{R_n}(h) = \frac{1}{n}\sum_{i=1}^n \1{h(X_i) \ne Y_i}
    \end{align*}

    \noindent Note that $\mathbb{E}[\widehat{R_n}(h)] = R(h)$ and $n\widehat{R_n}(h) \sim Binomial(n, R(h))$. In the following corollary of the Hoeffding's inequality, we will answer the question \textbf{how close the empirical risk is as an estimate of true risk} or \textbf{how fast the empirical risk converges to the true risk}.
\end{definition}

\begin{corollary}{Convergence of Empirical Risk}{convergence_of_empirical_risk}
    Given training data $\bigCurl{(X_i, Y_i)_{i=1}^n}$ such that each pair $(X_i, Y_i)\sim P_{XY}$ are independently identically distributed. Let $h:\mathcal{X}\to\mathcal{Y}$ be a classifier, we have:
    \begin{align*}
        P\bigRound{
            \bigAbs{\widehat{R_n}(h) - R(h)} \ge \epsilon
        } \le 2e^{-2n\epsilon^2}, \ \ \ \epsilon > 0
    \end{align*}
\end{corollary}

\begin{proof*}[Corollary \ref{coro:convergence_of_empirical_risk}]
    For all $1 \le i \le n$, we have $\1{h(X_i) \ne Y_i} \in \{0,1\}$. Hence, with probability one, $0 \le \1{h(X_i)\ne Y_i} \le 1$ and $b_i=1, a_i=0$ for all $1\le i\le n$.

    \noindent \newline Using the Hoeffding's inequality, we have:
    \begin{align*}
        P\bigRound{
            \bigAbs{\widehat{R_n}(h) - R(h)} \ge \epsilon
        } 
        &= P\bigRound{
            \bigAbs{\widehat{R_n}(h) - \mathbb{E}[\widehat{R_n}(h)]} \ge \epsilon
        } \\
        &= P\biggRound{
            \biggAbs{n\widehat{R_n}(h) - \mathbb{E}[n\widehat{R_n}(h)]} \ge n\epsilon
        } \\
        &\le 2\exp\biggRound{
            - \frac{2n^2\epsilon^2}{\sum_{i=1}^n (b_i - a_i)^2}
        } \ \ \ \text{(Hoeffding's inequality)} \\
        &= 2e^{-2n\epsilon^2}
    \end{align*}
\end{proof*}


\subsection{KL-divergence \& Hypothesis Testing}
\textbf{Set-up (Hypothesis Testing)} : Suppose that we have $\mathcal{Y}=\{0,1\}$ and $P_{XY}$ is a distribution on $\mathcal{X}\times\mathcal{Y}$. Let's assume that:
\begin{itemize}
    \item The prior probabilities $\pi_y$ are equal.
    \item The supports of likelihoods $p_0,p_1$ are the same.
    \item $0 < \alpha \le p_y(x) \le \beta < \infty$ for all $x\in \mathcal{X}$ such that $p_y(x)>0$ and for all $y\in\{0,1\}$.
\end{itemize}

\noindent Now suppose $X_1, \dots, X_n \sim p_y$ are independently identically distributed where $y\in\{0,1\}$ is unknown. Can we guess $y$ and how good our guess would be?

\begin{proposition}{KL-divergence hypothesis testing}{kl_hypothesis}
    From the above settings, the optimal classifier is given by the likelihood ratio test:
    \begin{align*}
        \widehat{h_n}(x) = \begin{cases}
            1 &\text{if } \frac{\prod_{i=1}^n p_1(x_i)}{\prod_{i=1}^n p_0(x_i)} \ge \frac{\pi_0}{\pi_1} \ \ (=1)
            \\ \\
            0 &\text{otherwise}
        \end{cases}
    \end{align*}
    \noindent Where $x=\bigRound{x_1, \dots, x_n}$ is an observation of the random vector $X=\bigRound{X_1, \dots, X_n}$. Define the class-specific risk $R_y(h)$ be the risk of misclassification when the true label is $Y=y$:
    \begin{align*}
        R_y(h) = P(h(X) \ne Y|Y=y)
    \end{align*}

    \noindent Then, we have:
    \begin{align*}
        R_0(\widehat{h_n}) \le e^{-2n D(p_0||p_1)^2 / c}, \text{ where } c = 4(\log\beta - \log\alpha)^2
    \end{align*}

    \noindent Where $D(p_0||p_1)$ is the $KL$-divergence of $p_1$ from $p_0$. We can prove a similar exponentially decaying bound for $R_1(\widehat{h_n})$.
\end{proposition}

\begin{proof*}{Proposition \ref{prop:kl_hypothesis}}
    We can rewrite the optimal classifier as:
    \begin{align*}
        \widehat{h_n}(X) = \begin{cases}
            1 &\text{if } \widehat{S_n}(X_1, \dots, X_n) \ge 0
            \\ \\
            0 &\text{otherwise}
        \end{cases} 
    \end{align*}

    \noindent Where we have:
    \begin{align*}
        \widehat{S_n}(X_1, \dots, X_n) &= \log \frac{\prod_{i=1}^n p_1(X_i)}{\prod_{i=1}^n p_0(X_i)} \\
        &= \sum_{i=1}^n \log\frac{p_1(X_i)}{p_0(X_i)} \\
        &= \sum_{i=1}^n Z_i \ \ \ \biggRound{\text{Letting } Z_i = \log\frac{p_1(X_i)}{p_0(X_i)}}
    \end{align*}

    \noindent Since the likelihoods are bounded, we have:
    \begin{align*}
        a_i = \log\frac{\alpha}{\beta} \le Z_i \le \log\frac{\beta}{\alpha} = b_i, \ \ 1 \le i \le n
    \end{align*}

    \noindent Now, we have:
    \begin{align*}
        R_0(\widehat{h_n}) 
            &= P(h(X) \ne Y | Y=0) \\
            &= P(\widehat{S_n} \ge 0 | Y = 0) \\
            &= P(\widehat{S_n} - \mathbb{E}[S_n | Y = 0] \ge - \mathbb{E}[S_n | Y = 0] | Y=0)
    \end{align*}

    \noindent To calculate the conditional expectation $\mathbb{E}[S_n | Y = 0]$, we have:
    \begin{align*}
        \mathbb{E}[S_n | Y = 0] &= n\mathbb{E}[Z_1|Y=0] \\
            &= n\int \log\frac{p_1(x)}{p_0(x)}p_0(x)dx \\
            &= -n\int \log\frac{p_0(x)}{p_1(x)}p_0(x)dx = -nD(p_0||p_1)
    \end{align*}

    \noindent Therefore, we have:
    \begin{align*}
        R_0(\widehat{h_n}) 
            &= P(\widehat{S_n} - \mathbb{E}[S_n | Y = 0] \ge nD(p_0||p_1) | Y=0) \\
            &\le \exp\biggRound{
                -\frac{2n^2D(p_0||p_1)^2}{\sum_{i=1}^n (b_i - a_i)^2}
            } \ \ \ (\text{Hoeffding's inequality})
    \end{align*}

    \noindent For every $1 \le i \le n$, we have:
    \begin{align*}
        b_i - a_i &= \log\frac{\beta}{\alpha} - \log\frac{\alpha}{\beta} \\
            &= \log\frac{\beta^2}{\alpha^2} = 2\log\frac{\beta}{\alpha} = 2(\log\beta - \log\alpha)\\
        \implies \sum_{i=1}^n (b_i - a_i)^2 &= 4n(\log\beta - \log\alpha)^2
    \end{align*}

    \noindent Finally, we have:
    \begin{align*}
        R_0(\widehat{h_n}) 
            &\le \exp\biggRound{
                -\frac{2nD(p_0||p_1)^2}{4(\log\beta - \log\alpha)^2}
            }
    \end{align*}

    \noindent Similarly, for $R_1(\widehat{h_n})$, we have:
    \begin{align*}
        R_1(\widehat{h_n}) 
            &\le \exp\biggRound{
                -\frac{2nD(p_1||p_0)^2}{4(\log\beta - \log\alpha)^2}
            }
    \end{align*}
\end{proof*}

\newpage
\subsection{End of chapter exercises}
\begin{exercise}{}{exercise_3.1}
    \begin{itemize}
        \item $\bf (i)$ Apply Chernoff's bounding method to obtain an exponential bound on the tail probability $P(Z\ge t)$ for a Gaussian random variable $Z\sim\mathcal{N}(\mu, \sigma^2)$.
        \item $\bf (ii)$ Appealing to the central limit theorem, use part $\bf (i)$ to give an approximate bound on the binomial tail. This should not only match the exponential decay given by Hoeffding’s inequality, but also reveal the dependence on the variance of the binomial.
    \end{itemize}
\end{exercise}

\begin{solution*}[Exercise \ref{ex:exercise_3.1}]
    .
    \begin{subproof}{$\bf (i)$ Chernoff's bounds for $Z\sim\mathcal{N}(\mu, \sigma^2)$}
        Using the Chernoff's bounding method, we have:
        \begin{align*}
            P(Z\ge t) 
                &\le \inf_{s>0}e^{-st}M_Z(s) \\
                &= \inf_{s>0}\exp\biggRound{
                    -st + \mu s + \frac{1}{2}\sigma^2 s^2
                } 
        \end{align*}

        \noindent The above bound is the tightest when the derivative of the term inside the exponential equals zero. Hence, we have:
        \begin{align*}
            -t + \mu + s\sigma^2 = 0 \implies s = \frac{t-\mu}{\sigma^2}
        \end{align*}
        \noindent From the above, we have the tightest Chernoff's bound as followed:
        \begin{align*}
            P(Z\ge t) \le \exp\biggRound{
                -\frac{(t-\mu)^2}{\sigma^2} + \frac{(t-\mu)^2}{2\sigma^2}
            } = \exp\biggRound{
                -\frac{(t-\mu)^2}{2\sigma^2}
            }
        \end{align*}
    \end{subproof}

    \begin{subproof}{$\bf (ii)$ Binomial tail upper bound}
        Let $S_n$ be the binomial random variable such that:
        \begin{align*}
            S_n = \sum_{i=1}^n X_i, \ \ X_i \sim Bernoulli(p)
        \end{align*}

        \noindent For a positive $\epsilon > 0$, we want to know the upper tail bound $P(S_n - \mathbb{E}[S_n] \ge\epsilon)$. Letting $\overline{X} = \frac{1}{n}S_n$, we have:
        \begin{align*}
            P(S_n - \mathbb{E}[S_n] \ge\epsilon) 
                &= P\biggRound{
                    \overline{X} - \frac{\mathbb{E}[S_n]}{n} \ge \frac{\epsilon}{n}
                }  \\
                &= P\biggRound{
                    \overline{X} - p \ge \frac{\epsilon}{n}
                }  \\
                &= P\biggRound{
                    \frac{\overline{X} - p}{\sqrt{pq}/\sqrt{n}} \ge \frac{\epsilon}{\sqrt{npq}}
                }, \ \ \ (q = 1-p)
        \end{align*}

        \noindent By the Central Limit Theorem, we have:
        \begin{align*}
            \frac{\overline{X} - p}{\sqrt{pq}/\sqrt{n}} \xrightarrow{d} \mathcal{N}(0,1)
        \end{align*}

        \noindent Hence, as $n\to\infty$, the upper tail bound would be:
        \begin{align*}
            P(S_n - \mathbb{E}[S_n] \ge\epsilon) 
                &= P\biggRound{
                    \frac{\overline{X} - p}{\sqrt{pq}/\sqrt{n}} \ge \frac{\epsilon}{\sqrt{npq}}
                } \\
                &\le \exp\biggRound{ -\frac{\epsilon^2}{2npq} } = \exp\biggRound{-\frac{\epsilon^2}{2Var(S_n)}}
        \end{align*}

        \noindent Double-check the bound with Hoeffding's inequality, we have:
        \begin{align*}
            P(S_n - \mathbb{E}[S_n] \ge\epsilon)  \le \exp\biggRound{-\frac{2\epsilon^2}{n}}
        \end{align*}
    \end{subproof}
\end{solution*}


\begin{exercise}{}{exercise_3.2}
    Can you remove the assumption in $0 < \alpha \le p_y(x)$? Consider other restrictions on $p_y$,
    other concentration inequalities, or other $f$-divergences.
\end{exercise}

\begin{solution*}[Exercise \ref{ex:exercise_3.2}]
    
\end{solution*}
